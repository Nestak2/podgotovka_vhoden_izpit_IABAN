% \documentclass{article}
\documentclass[a4paper,12pt]{article}
%Russian-specific packages
%--------------------------------------
\usepackage[T2A]{fontenc}
\usepackage[utf8]{inputenc}
\usepackage[bulgarian]{babel}
%Hyphenation rules
%--------------------------------------
\usepackage{hyphenat}
\hyphenation{ма-те-ма-ти-ка вос-ста-нав-ли-вать}
%-------------------------------------
% package below is for chem reactions
\usepackage{chemist}
\usepackage{isotope}
% --------------------------------------
% package for striking out
\usepackage{cancel}

%--------------------------------------
% force figure at desired position
\usepackage{float}

% make sections to begin on a new page
\usepackage{sectsty}
\sectionfont{\clearpage}

%for subfigures
\usepackage{subcaption}

%-----------------------------------
% I want to have less margins at the sides and top and bottom:
\usepackage[a4paper, total={6.5in, 9in}]{geometry}

% For angstroem symbol:
\usepackage{amsmath}
\let\oldAA\AA
\renewcommand{\AA}{\text{\normalfont\oldAA}}
%---------------------------------
% for degree symbol
\usepackage{gensymb}
%----------------
% for table of contents that is clickable
\usepackage{hyperref}
\hypersetup{
    colorlinks,
    citecolor=black,
    filecolor=black,
    linkcolor=black,
    urlcolor=black
}



\title{Развитие на конспектни теми за входен изпит при ИА-НАО\\
версия 1.0}
% \author{автори С.Г., Н.А.}
\date{Юни 2021}

\usepackage{natbib}
\usepackage{graphicx}

\begin{document}

\maketitle

\tableofcontents

%%%%%%%%%%%%%%%%%%%%%%%%%%%%%%%%%%%%%%%%%%%%%%%%%%%%%%%%%%%%%%
%%%%%%%%%%%%%%%%%%%%%%%%%%%%%%%%%%%%%%%%%%%%%%%%%%%%%%%%%%%%%%%
\section*{Пояснения за това четиво}
\addcontentsline{toc}{section}{Пояснения за това четиво}

Това четиво е направено като помощ за подготовка за входен изпит към Института по Астрономия в София. То предлага развитие на мнозинството от темите, но не всички. За някои теми на е писано нищо, за други е написано по-малко от колкото се очаква на изпита. Най-добре за развити темите относно галактична астрофизика, извънгалактична и космология. По темите относно общо естество на звездите съм си направил труда също да ги развия горе долу сносно. По темите относно по-странни звезди (променливи и избухливи) съм писал съвсем малко, ако въобще. Част от материала може да се пише в различни теми, например за качествата на спиралните галактики може да се пише в част Галактиката и в Галактиките, за космическото фоново лъчение може да се пише в част Разширение на вселената и в Еволюция на ранната вселена.\\

Осемнайсетте части на четивото отговарят на осемнайсетте теми от конспекта, подредени в същия ред. Всяка развита тема тук започва (в курсив) с точките, които са зададени за развиване на темата в официалния конспект.\\

Обемът писане за който трябва да се подготви един кандидат е към 4-5 страници на тема. На изпита се избират от комисията 2 теми. Кандидатът има 4 часа общо за развиването и на двете. Следва 30 мин почивка за оценяване на писмената част. След това устна част с отворен край. За писмената част е хубаво да се представят 2-3 скици и 2-3 уравнения, ако може, за устната също не е зле да се начертае скица/напише уравнение. Членовете на комисията по оценяване се отнасят колегиално, без ненужна дребнавост.\\

Върху написаното тук няма гаранция за вярност. Част от информацията е от учебници, част е от интернета, част е по спомен от лекции. Сигурен съм, че четивото ще помогне, но да бъде употребявано с доза мнителност. Също така писаното тук представлява сбора от бележките на няколко колеги.\\

нови версии и поправки на това четиво могат да бъдат намерени на \url{https://github.com/Nestak2/podgotovka_vhoden_izpit_IABAN/tree/main} Ако имате допълнения към някоя от недоразвитите теми, искате сами да допълните нещо в latex source file-а, намерили сте грешки или искате вид контакт ще се радвам да ми пишете през контакта представен там. В линка ще намерите и как може лесно и бързо сами да си свалите latex source file-а и сами да го обработвате/дописвате.\\

Успех!

%%%%%%%%%%%%%%%%%%%%%%%%%%%%%%%%%%%%%%%%%%%%%%%%%%%%%%%%%%%
%%%%%%%%%%%%%%%%%%%%%%%%%%%%%%%%%%%%%%%%%%%%%%%%%%%%%%%%%
\section*{Списък на съкращенията}
\addcontentsline{toc}{section}{Списък на съкращенията}

\begin{itemize}
\item МП - Млечен Път
\item ТДР - термодинамично равновесие
\item РЗК - Разсеяни звезди купове
\item ГП - главна последователност
\item ДХР - диаграма на Херцщрунг-Ръсел
\item ЗА - звездни асоциации
\item СЗК - сферични звездни купове
\item ХКГ - хоризонтален клон на гигантите
\item АКГ - асимптотния клон на гигантите
\item КЧГ - клон на червените гиганти
\item МС - междузвездна среда
\item ГМО - гигантски молекулярни облаци
\item MIR - mid infrared
\item СмЧД - свръх масивна черна дупка
\item АГЯ - активно галактично ядро
\item АГ - активна галактика
\item CMB - cosmic microwave background (също КМЛ)
\item КМЛ - космично микровълново лъчение (също CMB)
\item BAO - baryon acoustic oscillations
\item СЗЕ - Суняев-Зелдович-ефект
\end{itemize}


%%%%%%%%%%%%%%%%%%%%%%%%%%%%%%%%%%%%%%%%%%%%%%%%%%%%%%%%%%%%
%%%%%%%%%%%%%%%%%%%%%%%%%%%%%%%%%%%%%%%%%%%%%%%%%%%%%%%%%%%%%%%
\section*{Полезни уравнения}
\addcontentsline{toc}{section}{Полезни уравнения}

\begin{figure}[h!] %h!
\centering
\includegraphics[scale=.9]{polezni_uravneniq.pdf}
% \caption{}
\label{fig:SZE}
\end{figure}


%%%%%%%%%%%%%%%%%%%%%%%%%%%%%%%%%%%%%%%%%%%%%%%%%%%%%%%%%%%%%%
%%%%%%%%%%%%%%%%%%%%%%%%%%%%%%%%%%%%%%%%%%%%%%%%%%%%%%%%%%%%%
\section{Основни характеристики на звездите}
\textit{Зададени точки за развитие: Основни характеристики на звездите -  звездни величини, цветове, маси, температури, светимости, размери. Спектрална класификация.  Диаграма на херцшпрунг-ръсел (ДХР). Зависимост маса-светимост.}\\

\subsection{Основни характеристики на звездите}

- Масата е основна характеристика на една звезда. Масата определя темпа на ядрено  горене диктува пътя на звездната еволюция.  Температурата в даден слой от звездата зависи от масата, а оттам също и спектралната плътност $B_{\nu}(T)$ на лъчението и светимостта $L$ зависят от нея. Радиусът на Звездата $R$ също е функция на масата. Изброените параметри $T$, $B_0$, $L$ и $R$ също са основни характеристики на звездите.

- химичният състав е друго основна характеристика. В практиката често се използва величината металичност $[Fe/H] = lg(Z/Z_\odot)$, където Z  е съдържанието на метали (елементи с атомен номер по-голям от единица маса звезден газ), а $Z_\odot$  е съдържанието на метали за слънцето.

-  размерите на звездите се описват с техните радиуси. На главната последователност, ГП, (main branch)  едно M9-джудже има $R\approx 8 \cdot 10^{-2} R_\odot$, А една O8-звезда има $R\approx 10 \cdot R_\odot$.  Пак толкова имат гигантите (III),  свръх гигантите (I) имат $R\approx 2 \cdot 10 R_\odot$

- светимостите $L$ на здвездите вариират между $10^{-3} L_\odot$ и $10^6 L_\odot$\\



\subsection{Звездни величини}

 Лъчистият поток от звезда върху единица площ наричаме осветеност $E$ \footnote{https://en.wikipedia.org/wiki/Illuminance} $[lm/m^2 = lux]$.  В астрономията осветеността се измерва звездни величини. Традицията датира от времето на Хипарх (втори век преди новата ера). Предвид това и предвид факта, че човешкото око е логаритмичен приемник между звездната величина $m$  и осветеността $E$  съществува връзката

\begin{equation}
m = -2.512 \cdot lg(E) + const
\end{equation}

 ако две звезди създават осветеност съответно $E_1$ и $E_2$, като $E_1$ > $E_2$, то 

\begin{equation}
m_2 - m_1 = -2.512 \cdot lg(E_2/E_1) \; \Rightarrow \;  m_2>m_1
\end{equation}

 Последното уравнение се нарича закон на Погсън. Абсолютна звездна величина $М$ се нарича звездната величина на звездата, ако тя се намира на $d=10 pc$  от нас.  Тоест $E \sim d^{-2} \Rightarrow E_1/E_2=d_2^2/d_1^2$.  Акo $d_1=10 pc, \; d2=d$ то $E_{10pc}/E=d^2/100$

\begin{equation}
m-M=-2.512 \cdot lg(E/E_{10pc}) = -2.512 \cdot lg(100/d^2) = 5 \cdot lg(d) - 5
\end{equation}

$m-M$ се нарича модул на разстоянието\\

\subsubsection{Зависимостите на различните звездни качества}
Тук мога да опиша, че цвят -> $Т$, паралакс -> разстояние, разстояние и величина -> светимост, светимост и $T$ -> радиус. Алтернативно: цвят -> $Т$, широчина на линии -> класа, класа и $T$ -> светимост (през ДХР), светимост + величина -> разстояние

\subsubsection{Вогт-Ръсел теорем}
Знаейки масата и химичния състав на звездата можем да предскажем какво ще е нейното развитие, защото има само едно такова. В този смисъл масата и химичния състав са основните характеристики на звездите.


\subsection{Спектрална класификация, диагр. на Херцщпрунг-Ръсел}
Спектралните класи на звездите се описват според системата на Морган-Кийнан, от по-ярки към по-бледи, O, B, A, F, G, K, M и могат да бъдат запомнени с изречението ``Oh, be a fine girl, kiss me''\footnote{Това наименование на класи идва от двама астрономи в Харвард, които през 1890 подредили звездите според съдържанието на Н в спектрите - О най-много, В по-малко итн. След това техен асистент предложил да се запазят звездите в тези класи, но да се подредят по температурата - OBAFGKM. \url{https://astronomy.stackexchange.com/a/340/23399}}. Всяка група е подразделена с допълнително с числа от 0 до 9 (В0 е по-гореща от В9). Ако се направи графика на популация на звезди като на хоризонтала се отбележи температурата (или спектралната класа) и на вертикала се отбележи светимоста (или абсолютна яркост) се получава диаграма на Херцщпрунг-Ръсел. В тази диаграма се допълват с допълват с римски цифри класи на светимост. Така може да се различат в дадена класа дали звезда е джудже или гигант, например в класа М джудже може да се намира в подразделението МV, ярък гигант в МII, а свръх гигант в МIa. Пълната спектрална класа на слънцето е G2V, което обозначава, че е звезда на главната последователност (V) и че има повърхностна температура от близо 5800 К (G2).


\subsection{Цветове}
Звездите разполагат от една страна с визуален цвят, който отговаря физически на частта на спектъра в която е максимума на тяхната функция на Планк. Така например слънцето изглежда, че има (наблюдавана във вакума на свободния космос) жълт или жълто-бял цвят, защото с неговата повърхностна температура от 5800 К доминиращите електронно вълни на спектъра му са в този диапазон. От друга страна в астрономията съществува и качество на небесните тела което се нарича цветови индекс и е дефинирано като разлика между величините на звездата в 2 различни канала, най често разликата B-V. По-горещи и по-сини на цвят звезди имат цветови индекс B-V, който е по-малък (тъй като по-малка величина $m_B$ означава по-вече яркост в канал B), по-горещи и по-червени звезди поради същата причина ще имат по-голяма стойност на цветовия индекс B-V. Когато се изчислява цветовия индекс на звезда трябва да се вземе предвид и междузвездната екстинкцията, главно причинена от разсейването на светлината от частици прах, което кара звездите да изглеждат по-червени.  

\subsection{Зависимост маса-светимост}

На ГП $L \sim M^3$,  като за по-масивните звезди $L \sim M^2$.

\begin{figure}[h!]
\centering
\includegraphics[scale=0.5]{HRDiagram.png}
\caption{ДХР показваща главните качества на звездите, които могат да бъдат прочетени}
\label{fig:HRD}
\end{figure}
%%%%%%%%%%%%%%%%%%%%%%%%%%%%%%%%%%%%%%%%%%%%%%%%%%%%
%%%%%%%%%%%%%%%%%%%%%%%%%%%%%%%%%%%%%%%%%%%%%%%%%%%%

\section{Теория на звездните атмосфери}

\textit{Точки зададени за развиване:  Теория на Звездните атмосфери. Формиране на непрекъснатия спектър. Абсорбционни и емисионни линии. Профили на линиите. Механизъм на тяхното образуване и разширяване. Определяне на химичния състав на звездите}\\

 За да опишем лъчението ще са ни нужни няколко параметъра. \textbf{Интензитетът} на лъчението $I_{\nu}$ наричаме енергията $dE_{\nu}$ в честотен интервал $d\nu$, минаваща за време $dt$ през повърхност $d\sigma$ в рамките на пространствен ъгъл $d\omega$.  Ако $\phi$  и $\theta$ са ъглите на ориентация на $d\omega$ спрямо  нормалната $N$  към $d\sigma$:

\begin{equation}
I_{\nu} = \frac{dE_{\nu}} {cos(\theta) d\nu d\omega dt d\sigma}
\end{equation}


\begin{figure}[h!]
\centering
\includegraphics[scale=0.6]{geometriya_lachenie.png}
\caption{Геометрията на ъглите използвани за описване на лъчението}
\label{fig:geometriya_lachenie}
\end{figure}


\textbf{Поток} на лъчението $F_{\nu}$ наричаме лъчистата енергия в честотен интервал $d\nu$, преминаваща за единица време през единица площ направление на нормалната:

\begin{equation}
F_{\nu}d_{\nu} = \int I_{\nu} d\nu cos(\theta) d\omega = F_{\nu}^{+} + F_{\nu}^{-}
\end{equation}

 Това е съборът от изходящия ($F_{\nu}^{-}$)  и входящия ($F_{\nu}^{+}$) поток.

\textbf{Коефициентът на емисия} $j_{\nu}$ е излъчената енергия от обема $dV$ за единица време в честотен интервал $d\nu$ и пространствен ъгъл $d\omega$:

\begin{equation}
j_{\nu} = \frac{dE_{\nu}} {dt dV d\nu d\omega}
\end{equation}

\textbf{Коефициентът на абсорбция} $\chi_{\nu}$  е изменението на интензитета поради абсорбция в дължина $ds$:

\begin{equation}
dI_{\nu} = -\chi_{\nu} I_{\nu} ds
\end{equation}

\textbf{Оптична дебелина} $\tau_{\nu}$ се дефинира като

\begin{equation}
    \tau_{\nu} = \int_0^s \chi_{\nu} ds
\end{equation}

Уравнението на \textbf{лъчистия пренос} е

\begin{equation}
    dI_{\nu} = -\chi_{\nu} I_{\nu} ds + j_{\nu}ds
\end{equation}

В състояние на ТДР, когато на всеки елементарен процес на абсорбция отговаря такъв на емисия, на възбуждане отговаря релаксация и т.н., процесите са в равновесие във времето и интензитетът на лъчението се описва със закона на Планк и зависи само от температурата:

\begin{equation}
    I_{\nu}(T) = \frac{2h\nu^3} {c^2 (exp(h\nu / k_b T) -1)}
\end{equation}

 Звездите като цяло не са в ТДР - вътре в тях има горещ източник, извън тях Вселената е студена и има посока на енергопреноса. Въпреки това в малки обеми вътре в звездата условията са близки до ТДР, тоест лъчението идва отвсякъде - така наречено локално ТДР  (ЛТДР). Затова се оказва, че законът на Планк описва добре непрекъснатия спектър на звездите. Предпоставка да има във звездите ЛТДР, за разлика от други космически източници на лъчение като космически облаци например, е че в звездите плътността на материята е много по-голяма. Поради това частиците в звездите са в постоянен контакт с една друга, удрят се в една друга и това позволява да погълнат (и излъчат след това) най-различни стойности енергии в сравнение с дискретните енергии които частици в космически облаци поглъщат и излъчват. Тези най-различни излъчени енергии съставляват излъчването на черното тяло (функция на Планк).
 
 
\begin{figure}[h!]
\centering
\includegraphics[scale=0.2]{planck_function.png}
\caption{Функция на Планк и нейните апроксимации - Закон на Вийн за високи честоти и Рейли-Джийнс за ниски честоти}
\label{fig:planck_function}
\end{figure}

 
 \subsection{Абсорбционни и емисионни линии}

Просто правило е, че ако наблюдаваш светлина от горещ обект да минава през студен облак (и освен това и двете са на линията на наблюдение!) се виждат абсорбционни линии, а ако се наблюдава горещ облак, който не е в ТДР, се виждат емисионни линии \footnote{\url{https://astronomy.stackexchange.com/questions/43609} , \url{https://www.e-education.psu.edu/astro801/content/l3_p6.html}}. Абсорбционни линии в звездния спектър се наблюдават, когато по лъча на зрение повече фотони в дадена честота се поглъщат, отколкото се излъчват. Това може да се дължи на фото йонизация (свързано - свободни преходи) при $\nu > \nu_{ion} = E_{ion}/h$, нa свързано - свързани преходи и при $\nu = \nu_{ij}=E_i - E_j$, на фото дисоциация на молекули (при хладните звезди). Или може да се каже по-просто, че абсорбционните линии се дължат на това, че над горещите слоеве на звездата има по-студена фотосфера - част от постоянни спектър на звездата бива поглъщан на дискретни честоти от фотосферата и после излъчен в случайни посоки.

Както написах по-горе, емисионни линии се получават, когато най-външният слой на звездата е по-горещ от долните. В тази среда атомите са във възбудени състояния, но и постоянно падат към по енергийно ниски състояния и излъчват дискретни честоти.

\subsection{Профили на линиите}

 Спектралните линии имат профили, определение от комбинацията от законите на квантовата механика и от свойствата на газа. Първо, ``лабораторна'' абсорбционна линия няма да има профил на $\delta-$ функция заради принципа на Хайзенберг: $\Delta \lambda \Delta t \geq \hbar / 2$;  за $H \alpha$ $\Delta \lambda \approx 10^{-4}$\AA.  Това разширение на линията носи името \textbf{ естествена ширина на линията} и има Лоренцов профил.

Спектралните профили имат също \textbf{Доплерово разширение} заради a) кинетичната енергия на частиците, което има температурен характер, б) околоосното въртене на Звездата. По-висока температура и по-високо $v_{rot} \cdot sin(i)$ - по-голямо разширение. За $T=5800 K$ се получава $\delta \lambda_{H\alpha}=0.4$\AA. Това разширение има Гаусов профил.

 Налице е също \textbf{ударно разширение} на линиите: частици, които са в процес на преход се повлияват от други, ако те са достатъчно близки. Тези удари водят до скъсяване на времето, което прекарват в това състояние и по-този начин резултата е сходен на естественото разширение. Сборният ефект от всички такива взаимодействия се наблюдава като разширение, степента на разширение зависи от $\rho$ и $g$: по-високо $g$ води до повече разширяване. Затова можем по спектъра на една звезда да оценим светимостта й. Профилите на линиите се променят и при наличие на електрично или магнитно поле (ефекти на Щарк и Зееман, съответно).

 Резултатният профил на линията се описва с функцията на Voigt.

\subsection{Определяне на химичния състав на звездите}

 Важен инструмент за определяне на химичния състав на звездите е \textbf{кривата на нарастване} - зависимостта $W_{\lambda}(logN)$, а където $N$ е броят поглъщащи атоми, а $W_{\lambda}= \int \frac{I_0 - I_2} {I_0} d\lambda$ се нарича \textbf{еквивалентна ширина на линията}.

 Ако в атмосферата на една звезда няма определен химически елемент, неговите линии също няма да ги има. Ако добавим малко количество ще получим слаба абсорбционна линия, за която $\tau<1$ (оптично тънка). Ако броят атоми се удвои, броят погълнати фотони също ще се удвои и $W_{\lambda}$ също. Така за малки $N$ се получава $W_2 \sim log(N)$. С по-нататъшното увеличаване на $N$ $\tau$ може да надхвърли 1 и тогава максималният брой фотони с честота $\lambda$ бива погълната. С добавянето на още фотони дъното се насища; крилата все още са оптично тънки и се задълбочават: $W_{\lambda} \sim \sqrt{logN}$. При още повече атоми доминира ударното разширение и $W_{lambda} \sim \sqrt{N}$.

 С помощта на кривата на нарастването можем да определим броя поглъщащи атоми; после с уравненията на Болцман и Саха можем да сметнем броя на атомите на този химичен елемент във всички състояния над фотосферата. За да се намали грешката е добре да се използват няколко линии на един и същи елемент в еднаква начална орбитала.

%%%%%%%%%%%%%%%%%%%%%%%%%%%%%%%%%%%%%%%%%%%%%%%%%%%%%%%%%%%%%%%%
%%%%%%%%%%%%%%%%%%%%%%%%%%%%%%%%%%%%%%%%%%%%%%%%%%%%%%%%%%%%%%%
\section{Вътрешен строеж на звездите}

\textit{Точки за развитие по темата: Вътрешен строеж на звездите. Условия за механично равновесие. Източници на звездната енергия. Лъчист и конвективен пренос на енергия.}

\subsection{Вътрешен строеж на звездите}

Най-общо, звездите са изградени от ядро, обвивка, външният граничен слой на която се нарича фотосфера, атмосфера\footnote{доста полезно тук: \url{wikipedia.org/wiki/Stellar_structure}}. Поради температурите и наляганията вътре в звездата, веществото е в състояние на плазма (освен при изродените обекти). Ядрото е този централен обем от звездата, където протичат ядрените реакции по горене на водорода (p-p или CNO) по време на етапа на главна последователност (ГП).  Радиално навън следва околоядрената обвивка. Дали тази обвивка е лъчиста или конвективна зависи от градиента на температурата (голям градиент => конвективен пренос) и от оптичната плътност (висока оптична плътност => конвективен транспорт). Има грубо правило, че звезди с $М>1.5M_{\odot}$ имат конвективна вътрешна обвивка и лъчиста външна, звезди с $1.5M_{\odot}< M_{\odot} < 0.5M_{\odot}$ имат лъчиста вътрешна обвивка и конвективна външна, а път звезди с $M < 0.5M_{\odot}$ са изцяло конвективни. За фотосфера често се приема повърхнината, за която оптичната дебелина $\tau=2/3$  за $\lambda=5500$ \AA. ($d\tau_{\nu}=\chi_{\nu} \rho ds$, където $\chi$ e коефициент на непрозрачност, $\rho$ е плътност, $ds$ е линеен интервал; $\chi_{\nu}$ се въвежда като $dI_{\nu}^{-}=-\chi_{\nu}\rho I_{\nu} ds$, $I_{\nu}$ е интензитетът на лъчението). Състоянието, размерите и характеристиките на ядрото, обвивката и атмосферата силно зависят от масата и еволюционния стадий на звездата.

\begin{figure}[h!]
\centering
\includegraphics[scale=0.2]{Heat_Transfer_in_Stars-en.svg.png}
\caption{Състав на звездите}
\label{fig:Heat_Transfer_in_Stars-en}
\end{figure}

\subsection{Условия за механично равновесие}

 Звездата е в механично или хидростатично равновесие тогава, когато гравитационната сила, действаща към центъра й, бива уравновесена от вътрешното налягане навсякъде в звездата:

\begin{equation}
\frac{dp(R)} {dr} = -\frac{GM_R \rho(R)} {R^2}
\end{equation}

където $M_R = \int_0^R M(R^{'})dR^{'}$



\subsection{Източници на звездната енергия}

По време на етапа на ГП и последващите еволюционни стадии звездите излъчват енергия посредством реакции на ядрен синтез. В хода на реакциите по-леки атомни ядра се сливат, образуват по-тежки, разликата $\Delta m = m_{end} - m_{begin}$ (масов ефект)  се превръща в енергия: $E=\Delta m c^2$. По време на ГП това са реакции на сливане на водород в хелий - p-p верига или CNO-цикъл според масата на звездата. По-следващите етапи на звездната еволюция синтезират по-тежки елементи: He -> C/O за звездите $M<8M_{\odot}$ или до Fe за $M>8M_{\odot}$.

 Реакциите на синтез протичат в ядрото по време на ГП, в по-късните етапи на еволюцията може да има синтез и в обвивката. По време на етапа на звездообразуване физическите условия в Звездните недра все още не позволяват синтез. Тогава звездата отделя енергия за сметка на гравитационната потенциална енергия. Ако цялата гравитационна енергия на контракция/падащ върху протозвездата материал $\dot{M}$ се превръща в лъчиста енергия $L$, то

\begin{equation}
L=\frac{GM\dot{M}} {R}
\end{equation}

Такъв вид производство на лъчение от звездата от свиване се нарича Келвин-Хемхолц механизъм с времево развитие $t_{kn}=1/2 \frac{GM^2} {RL}$ (за слънцето това е $\approx 10^7$ yr). В действителност ядреният синтез, който съществува в момента в слънцето е доста по-ефективен от Келвин-Хелмхолц механизъма, което може да се разбере и от факта, че всъщност слънцето е далеч по-възрастно от $10^7yr$ и ще синтезира водород приблизително $10^{10}$ yr.

\subsection{Лъчист и конвективен пренос на енергия}

условието за лъчисто равновесие гласи, че отделената в ядрото енергия трябва да се излъчи:

\begin{equation}
\frac{dL(R)} {dR}= 4\pi R^2 \rho(R) \epsilon(R)
\end{equation}

като $\epsilon$ е темпа на енергоотделяне. Има три начина топлинната енергия да се пренася: конвекция, лъчеизпускане и топлопроводност. Последната представлява обмяна на кинетичната енергия между сблъскващи се частици и играе важна роля в крайните етапи на звездната еволюция - изродените обекти. Лъчеизпускането е процес при който частиците намаляват вътрешната си топлинна енергия за сметка на излъчване на фотони при преминаване в по-ниско енергично състояние. Конвекцията пък представлява крупно мащабно движение на т.нар. Конвективни потоци и в течности, които пренасят частици с по-висока или по-ниска температура до области от течности с по-ниска или по-висока температура съответно.


%%%%%%%%%%%%%%%%%%%%%%%%%%%%%%%%%%%%%%%%%%%%%%%%%%%%%%%%%%%%%%%%%%%%
%%%%%%%%%%%%%%%%%%%%%%%%%%%%%%%%%%%%%%%%%%%%%%%%%%%%%%%%%%%%%%%%%%%%
\section{Еволюция на звезди с различни маси}
\textit{Точни за развитие на темата: Еволюция на звезди с различни маси. Ранни етапи на еволюция. Еволюционни трекове на ДХР. Късни етапи в звездната еволюция - бели джуджета, неутронни звезди и черни дупки}


\subsection{Ранни етапи на еволюцията}
Звездите се създават в междузвездни облаци от газ и прах. За да се създадат звездите тези облаци тябва да минат през процеси на сгъстяване и фрагментиране при които облаците обикновено се свиват/колабират в множество райони. Това за да се случи трябва да е изпълнен критерият на Джинс, който включва че хидростатичното равновесие

\begin{equation}
    dP = g\rho dR
\end{equation}

трябва да е нарушено с гравитацията доминираща над налягането. Джинс дава граница на минимални маса и радиус на район в облака, за дадени температура и плътността на газ, при които гравитацията ще надделее над налягането и района ще колабира. Тези граница могат да се извлекат от прилагането на виралната теорема и че потенциалната енергия на гравитацията трябва да доминира над кинетичната енергия на газа, която му дава налягане навън

\begin{gather}
    E_k < \frac{1}{2}E_p\\
    \frac{3}{2}NkT < \frac{1}{2}\frac{GM^2}{R}\\
    3NkT < \frac{GM^2}{R}
\end{gather}

Тук може да приложим $M=N\mu$ и $M=4/3 \pi \rho R^3$:

\begin{gather}
    \label{eq:jeans_derivation}
    3\frac{\cancel{M}}{\mu}kt < \frac{GM^{\cancel{2}}}{R}\\
    3\frac{kT}{\mu} < \frac{G}{\cancel{R}} \frac{4}{3}\pi R^{\cancel{3}} \rho\\
    \left( \frac{9}{4} \frac{kT}{\mu G\pi \rho} \right)^{1/2} < R
    \label{eq:R_jeans}
\end{gather}

като граничният радиус при който \eqref{eq:R_jeans} е действително уравнение се нарича радиус на Джинс $R_j$. По сходен начин може да се изведе и масата на Джинс $M_j$, като в неуравнение $\eqref{eq:jeans_derivation}$ се изрази не масата като функция на радиус, ами обратното, $R=(\frac{3M}{4\pi \rho})^{1/3}$:

\begin{gather}
    3\frac{kT}{\mu} < GM \left( \frac{4\pi \rho}{3M} \right)^{1/3}\\
    3\frac{kT}{\mu} \left( \frac{3}{4\pi \rho} \right)^{1/3} < GM^{2/3}\\
    \left( \frac{3kT}{\mu G} \right)^{3/2} \left( \frac{3}{4\pi \rho} \right)^{1/2} < M
\end{gather}

С тези изисквания, ако район на газов облак изпълни неуравненията и има маса или радиус по-големи от посечените, гравитацията ще надвие налягането и ще започне свиването на района.\\

Докато газът се свива той развива и въртящо движение около една ос. Това е естествено за всички космически тела които по един или друг начин минават през процес на колапс. Поради запазването на момента на импулса $\Vec{L} = m\Vec{v} \times \Vec{r}$ това води до факта, че с продължително свиване и скъсяване на $\Vec{r}$ скоростта $\Vec{v}$ трябва да се увеличи, което от своя страна води до увеличаване на радиалната сила на въртенето $F=mv^2/r$ която е обратно насочена на гравитационната сила и това забавя свиването на района. Тангенциалната скорост $v$ е най-висока по екватора на въртящия се район и по отслабва с увеличение на ъгловото отстояние от екватора. Поради това части на газа, които не са на екватора колабират по бързо, като придават с времето все по ``сплескана'' по полюсите форма на района докато не се оформи диск. При продължаващото свиване на диска, при което $v$ продължава да се увеличава, ако $v$ достигне орбиталната скорост, тогава приближаването на тази част от района към центъра спира напълно. Една част от масата на облака и диска все пак акретира върху ядрото и светимостта $L$ от този процес може да се извлече по следния начин: 

\begin{gather}
    E_p = \frac{GMm}{r}\\
    L = \frac{dE_k}{dt} = \frac{dE_p}{dt} = \frac{GM dm}{rdt} = \frac{GM}{r} \dot{m}
    \label{eq:L_acretion}
\end{gather}

Горното уравнение \eqref{eq:L_acretion} важи, ако се приеме, че всичката енергия от акретирана маса се излъчва обратно и не се задържа от ядрото на свиващият се и да води до увеличение на температурата на ядрото.\\ 

Ядрото на района също се свива. За него, както и за обграждащият го газ, важи правилото, че за да може да се свива газът трябва да е прозрачен, в смисъл, че трябва да може да позволява излъчването на енергия чрез лъчение, иначе свиването не би било възможно. С напредване на времето ядрото става все по-гъсто и по някое време става непрозрачно за видимото излъчване, което противодейства на свиването. Ако ядрото е натрупало достатъчно маса свиването ще продължи и ядрото ще излъчва лъчение по механизма на Келвин-Хелмхолц. В тази фаза на развитието си ядрото представлява прото звезда. Ако ядрото продължи да трупа маса и тя доведе до достатъчно големи налягания и температури се стига до ядрен синтез на деутериум. Когато лъчението на прото звездата стане толкова силна, че то издуха останалия газ около нея и по този начин спре натрупването на маса, прото звездата преминава във фазата на звезда преди главната последователност (ГП). В тази фаза светимостта на тялото е приблизително константна, но температурата на повърхността му се покачва. Когато налягането в центъра на тялото стане достатъчно голямо и температурата стане приблизително $10^{11}$ K звездата преди главната последователност започва да синтезира водород и с това преминава във фазата на главната последователност. С това спира и свиването на центъра на звездата, както и увеличаването на повърхностната ѝ температура. Звездата навлиза в най-дългата фаза от ядрено активния си живот. 


\subsection{Еволюция на звезди с разлини маси}

\subsubsection{Еволюционни трекове на ДХР}
Звездите преди ГП следват с времето еволюционни трекове в ДХР, които зависят от тяхната маса и химичен състав. Такива трекове се наричат Хаяши трекове и са представени във фиг. \ref{fig:evol_tracks_stars}. За звезди с $M<\approx 0.5 M_\odot$ трекът представлява вертикална линия в ДХР и тази вертикална линия се нарича Хаяши трек. Еволюционните трекове на звездите преди ГП свършват с навлизането на телата в ГП. За протозвезди с $M > 0.5 M_\odot$ само първоначално трекове са Хаяши трекове (с времето $L$ спада, но $T_{eff}=const$), но след някакво време навлизат в Хейней трек, който представлява хоризонтала в ДХР ($L=sonst$ докато $T_{eff}$ се покачва). Преходът от Хаяши режим в Хайей режим става когато напълно конвективната звезда преди ГП развие лъчист слой в строежа си. Звезди преди ГП с $M> \approx 3 M_\odot$ не се развиват по Хаяши режим ами следват изцяло режима на Хейес.\\

При звездите преди ГП има и друго разделение основано на масите - тези звезди се класифицират в две клас, Т Таури и Ае/Be звезди на Хербиг. Т Таури звездите са звезди преди ГП, които са на възраст $<10^6$ yr. Те показват силна променливост във величина и силни хромосферни линии. Т Таури звезди имат маси $<M_\odot$, температури сходни с тези на ГП, но много по-големи яркости. Последното се дължи на факта, че Т Таури звездите са все още в просец на свиване и имат по-големи размери от звездите на ГП. Голяма част от светимоста на Т Таури звездите идва от потенциалната енергия на акретиращата маса, но за по-късните етапи на развитието на тези звезди, се стига до синтез на ядра по-тежки от водород, включително и на литий. По масивните аналози на Т Таури звездите се наричат Ae/Be Хербиг обекти и имат маси $2-8 M_\odot$. 


\begin{figure}[h!]
\centering
\includegraphics[scale=0.6]{evol_tracks_stars.png}
\caption{Еволюционни трекове на звездите преди ГП за разлъчни маси}
\label{fig:evol_tracks_stars}
\end{figure}



\begin{figure}[h!]
\centering
\includegraphics[scale=0.5]{HRD_movement.png}
\caption{Състав на звездите}
\label{fig:HRD_movement}
\end{figure}

\subsubsection{Късни етапи на развитие - маломасивни звезди}
Звездите прекарват по-голямата част от съществуването си в ГП. Приблизително дължината на съществуването може да се изчисли имайки предвид, че тя е пропорционална на наличния запас от маса за ядрена енергия ($\sim M$) и е обратно пропорционална на светимоста ($\sim 1/L$). За по-малко масивните звезди на главната последователност важи $L \sim M^{3.5}$. Знаейки приблизително колко време ще прекара слънцето на ГП може да изразим дължината на съществуването на друга звезда от ГП като отношението

\begin{equation}
    \frac{\tau_{sMS}}{\tau_{\odot MS}} = \frac{M_s}{M_\odot} \frac{L_\odot}{L_s} = \frac{M_\odot^{2.5}}{M_s^{2.5}} 
\end{equation}

По малосавините звезди ($M<3M_\odot$) стигат точна на обръщане и напускат ГП когато водорода в ядрото е почти изчерпан и водородно горене има само във външен слой на ядрото. Ядрото се свива, външната обвивка се разширява, $T_{eff}$ спада, светимоста остава почти константна. Обвивката става напълно ковективна. Звездата се катери по линията на Хаяши и от субгигант става червен гигант.\\

При звезди с $M<0.5M_\odot$ развитието на звездата спира с това, че водорода се изчерпва в слоя около ядрото и после звездата просто изстива. При по-масивни звезди ядрото продължава да се свива до $10^8$ К. Тогава се стига до ``хелиевото избухване'' - рязко настъпващ синтез на He в O и C. Свиването на ядрото спира и звездата се намира на \textbf{хоризонталния клон}. Това спира, когато звездата натрупа в ядрото толкова O и C, че хеливият синтез се случва също само в слой, под слоят на H. Тогава ядрото започва пак да се свива, външните обвивки да се разширяват и звездата се намира на \textbf{асимптотния клон на гигантите, AGB}. Когато слоят на горене на H се изчерпа външната H обвивка започва периодично да се свива и разширява с което да снабдява нов H върху слоя на He синтез с което отново започва H да се синтезира и този процес се повтаря. Процеса се нарича \textbf{термични пулсации}. При тези осцилации се стига до свръх вятър, който издухва обвивката на звездата - създава се \textbf{планетарна мъглявина}. Оголеното C/O ядро се превръща в \textbf{бяло джудже}.



\begin{table}[]
% \begin{tabular}{c|c|c|c|c|c|c|c}
\begin{tabular}{p{0.18\linewidth}|p{0.30\linewidth}|p{0.08\linewidth}|p{0.06\linewidth}|p{0.06\linewidth}|p{0.06\linewidth}|p{0.06\linewidth}}
тяло               & настъпва когато                                                          & $t$ [yr]     фаза & $M/M_\odot$ & $T_{eff}$ [K]     & $T_c$ [K]           & $R/R_\odot$  \\
\hline
\hline
мол./газ облак     & свиване на студен газ                                                    & -                 & -           & 10                & 10                  & $10^8$       \\
\hline
прото звезда       & свиване, излъчване на гравитационна $E$; после деутерриум горене         & $10^5$            & 1           & 100               & $10^4-10^6$         & 100          \\
\hline
зв. преди ГП       & издухване и $\dot{M}=0$, но има диск; синтез на метали; Т Таури и Хербиг & $10^6 - 10^7$     & 1           & 4000-6000         & $10^6-5 \cdot 10^5$ & 10           \\
\hline
ГП                 & синтез на H; свиване спира                                               & $10^{10}$           & 1           & 4000-6000         & $1.5 \cdot 10^7$    & 1            \\
\hline
червен (суб)гигант & H горене само в слой - ``точка на обръщане''; свиване ядро и 3$\alpha$   & $10^9$            & 1           & 4000-6000         & $5-30 \cdot 10^7$   & 3-500        \\
\hline
планет. мъглявина  &                                                                          & $10^5$            & 0.4         & 3000              & -                   & 1000         \\
\hline
бяло джудже        &                                                                          & $>10^10$          & 0.6         & 100               & 50000               & 0.01     
\end{tabular}
\caption{качества на различните фази от развитието на маломасивна звезда ($M<3M_\odot$)} 
\end{table}

\subsubsection{Късни етапи на развитие - масивни звезди}
В масивните звезди ($M>3M_\odot$) ядрото не е толкова гъсто колкото в маломасивните и заради това масивните звезди не преминават през много фази на електронно израждане. С привършването на H в ядрото и масивните звезди се разрастват и изстудяват, но светимоста им остава почти константна. Звезди с маси $>25M_\odot$ и с висока металичност почват да губят външната си обвивка заради силни ветрове още докато синтезират водород. Тези звезди се наричат Волф-Райе звезди. Масивните звезди изгарят различни елементи в слоеве на ядрото и колкото по-масивна е звездата толкова по тежки елементи може да синтезира в ядрото. Най-тежкият елемент, който може да се генерира в звездите е желязо.\\

Когато се изчерпи горивото за ядрен синтез в тези слоеве ядрото на масивна звезда подлежи на много бързо свиване. При него се стига до неутронно изразждане в най-вътрешната част на ядрото - протони и електрони са синтезирани в неутрони, които са наредени плътно един до други и не позволяват постепенно по-нататъшно свиване при постепенно увеличаване на налягането върху ядрото (все пак при огромни налягания е възможно свиване до черна дупка). Пора това неутронно израждане падащите слоеве към ядрото се компресират ударно, нагряват се и се стига до мощно избухване. Това избухване се нарича \textbf{свръх нова}, то разкъсва слоевете над ядрото и има за кратък момент светимост от $10^10 L_\odot$. Главната част на излъчената енергия обаче е под формата на неутрината, което се дължи на ядрени реакции от сорта на бета разпад, фотодезинтерграция и сливане на протони и електрони. По време на свръх новата се стига и до ядрените реакции които създават много тежки ядра, които после биват разпилени в околностите и са част от развитието на следващото поколение звезди. Голямата част от елементите по-тежки от желязо се формират по време на свръх нова.

\subsubsection{Бели джуджета}
Белите джуджета са остатъци от средномасивни звезди с $M<8M_\odot$. Те са плънти и горещи тела (спектрални класи B-F) с радиус приблизително колкото земния. Голямата им плътноста се дължи на факта, че поради огромната гравитационна сила са смачкани до малки размери преди да се разделят с обвивката на звездата и в това състояние налягането в бялото джудже не е с термодинамичен произход ами с квантово механичен. Може да се каже, че енергията на Ферми доминира над топлинната енергия в бяло джудже. Може да се каже и че налягането съпротивляващо се на гравитацията в бяло джудже произлиза от принципа на несъвместимост на Паули, който забранява в един атом 2 електрона да бъдат идентични във всички квантови числа (да заемат еднакви квантови състояния). Въпреки това, бялото джудже може да поеме повече маса и забележителното е, че поради ефектите на квантовата механика по-масивни бели джуджета имат по-малки радиуси. Белите джуджета разполагат с горна граница за маса която могат да достигнат $M_{ch} \approx 1.44 M_\odot$. Приблизителността в тази граница се дължи на това, че релативистични ефекти, например от околоостното въртене, влияят на тази граница, наричана маса на Чандрасекар. Ако бялото джудже се намира в двойна система може да акретира маса от спътника. Ако при това се надхвърли масата на Чандрасекар бялото джудже или ще колабира до неутронна звезда чрез синтез на протони и електрони или ще се стигне до мощни и бързи ядрени реакции които ще доведат то свръх нова от тип 1а при която бялото джудже ще бъде разкъсано. СН тип 1а са най-големият източник на желязо в галактиките.


\subsubsection{Неутронни звезди}
Неутронните звезди се зараждат при колапса на по-масивни звезди при които електронното израждане не може да се съпротиви на гравитационната сила на свиващото се ядро. Ефекта на това е, че електрони и протони се сливат в бърз процес съпроводен от излъчването на неутринота ($p + e^- \rightarrow n + \nu_e$). Резултата от това е, че неутронните звезди са изключително плътни, тъй като радиуса на електронния облак е приблизително $10^5$ от радиуса на ядрото на атом и практически в неутронна звезда цялата маса е свита до едно огромно атомно ядро. Налягането на неутронните звезди също е диктувано от принципа на Паули, сходно на налягането в бели джуджета. Типични маси на неутронни звезди са $\approx 1-2 M_\odot$, а радиусите са $\approx 10-100$ km. Горната граница за масата на неутронни звезди е $\approx 2.8 M_\odot$ (граница на Опенхаймер-Волков) и е наложена от квантово механични ефекти. Друго често срещано качество на неутронните звезди е бързото им околоостно въртене. То се дължи на консервация на момента на импулса и на голямото свиване на звездата, която преди това се е въртяла по-бавно. Високото околоостно въртене е наблюдавано посредством специален вид неутронни звезди - пулсарите. Това са неутронни звезди, чиито магнитни полюси, поради въртенето на неутроните звезди, периодично сочат към земята и в тези моменти засичаме силни радио сигнали, но също така сигнали и в други части на спектъра. Честотите с които неутронни звезди се въртят са стигат до порядъка на секунди до десетки милисекунди.

\subsubsection{Черни дупки}
Ако масата на централното тяло надхвърли границата на Опенхаймер-Волков неутроновото израждане не може да предотврати последващо свиване на тялото. Такива тела добиват размери при които гравитационното ускорение $g$ около тях (или изкривяването на пространство времето) е толкова голямо, че не позволява на светлината да напусне тялото. Поради това, тези тела биват наричани  черни дупки. Радиуса под който трябва да се свият такива тела, или алтернативно, разстоянието от такова тяло при което дори и светлината не би могла да се измъкне от гравитационната сила може да се изчисли сравнително лесно. Основата на изчислението е, че при такъв радиус $r_s$ скоростта за избягване от гравитационното поле е равна на $c$. Използвайки уравнеието за кинетичната и гравитационната енергия получаваме:

\begin{gather}
    E_{kin} = E_g\\
    \frac{1}{2}mv^2 = \frac{GMm}{r}
    \Rightarrow r = \frac{2GM}{v^2}\\
    v=c \rightarrow r_s = \frac{2GM}{c^2}
\end{gather}

Този радиус е изчислен за пръв път от Шварцшилд и носи неговото име. За пример, Шварцшилд радиуса на слънцето би бил 3 км. По дефиниция такива тела не могат да бъдат наблюдавани, тъй като електромагнитно лъчение не може да ги напусне. Но наблюдения на тела обикалящи черни дупки, лъчението на АГЯ и наблюдения на гравитационни вълни са в много добро съглашение с теоретични маси на черни дупки за такива централни тела. Около черна дупка се стига и до значими ефекти на ОТО, като например червено отместване на светлина поради гравитационната сила, ефекти на въртяща се черна дупка (черна дупка на Кер) и изпаряване на черни дупки (последното се дължи на ефекти от квантовата механика). Черните дупки нямат теоретична максимална маса. Най-масивните от тпх се намират в активни галактични ядра (АГЯ) и трупат маса от акреция на материя. Изчисления на лъчението на АГЯ сочат, че най масивните свръх масивни черни дупки са от величината на $10^10 M_\odot$.




\subsubsection{Развитие на радиуса на звезда}

\begin{figure}[h!]
\centering
\includegraphics[scale=0.5]{star_rad_evol.png}
\caption{Състав на звездите}
\label{fig:star_rad_evol}
\end{figure}


%%%%%%%%%%%%%%%%%%%%%%%%%%%%%%%%%%%%%%%%%%%%%%%%%%%%%%%%%%%%%%%%%%%
%%%%%%%%%%%%%%%%%%%%%%%%%%%%%%%%%%%%%%%%%%%%%%%%%%%%%%%%%%%%%%%%%%%

\section{Двойни и кратни звезди}

\textit{Двойни и кратни и звезди. Визуални, спектрални и фотометрични двойни системи. Образуване. Определяне на параметрите на една двойна система.}\\

\subsection{Визуални, спектрални и фотометрични двойни системи}
Звезди, които са достатъчно близо една до друга, за да си влияят гравитационно, се наричат двойни или кратни (ако членовете на системата са повече от два) звезди. Някои звезди изглеждат двойни/ кратни при наблюдения от земята единствено защото са разположени близо по лъча на зрение. Ако между тях в действителност няма гравитационна връзка, то тези системи се наричат \textbf{оптично двойни}. Ако при наблюдения компонентите на системата видимо са отделени една от друга и наблюдателно можем да ги разграничим, то системата е \textbf{визуално двойна}. В някои системи отделните компоненти не се разграничават визуално при наблюдения, но в процеса на орбитиране около общия център на масите се наблюдават изменения (периодични) в кривата на блясъка на системата в следствие периодичното затъмняване на компонентите една от друга. Тъй като такива системи проявяват фотометрична променливост те се наричат \textbf{фотометрични двойни}.  Когато двойнственият характер на системата се наблюдава само по доплеровото отместване на спектралните линии на различните компоненти в следствие различните им във времето скорости на движение спрямо наблюдателя, системата се нарича \textbf{спектрално двойна}. По отношение на разстоянието между компонентите си, системите биват широки, ако са на достатъчно разстояние една от друга, за да се развиват като единични звезди и тесни, ако близостта им влияе на хода на развитието им.

\subsection{Определяне на параметрите на една двойна система}
Движението на компонентите на двойните системи се случва по законите на Кеплер. Всяка компонента описва елиптична орбита около общия център на масите. Ако приемаме едната компонента за неподвижна, то другата ще орбитира около нея по елиптична орбита. Изброените три орбити са подобни - имат еднакъв ексцентрицитет. Голямата полуос $а$ на спътника около главната компонента е равна на сумата $a_1 + a_2$ от големите полуоси на орбитите на двете компоненти около общия център на масите. Ако от наблюденията ни е известна $а$, то от третия закон на Кеплер може да сметнем

\begin{equation}
    M_1 + M_2 = \frac{4\pi^2 a^3}{GT^2}
\end{equation}

Ако ни е известно отношението $a_2/a_1$ можем пак по третия закон да определим

\begin{equation}
    \frac{M_1}{M_2} = \frac{a_2}{a_1}
\end{equation}

Огромната важност на двойните системи в астрономията е, че ни позволяват да пресметнем най-важната характеристика на звездите - тяхната маса. При фотометрично двойните на основа кривата на блясъка можем да определим ъгъла на орбитата $i$ спрямо лъча на зрение, $R_1/R_2$ по продължителността на минимума, $L_1/L_2$ по дълбочината на минимума (ако затъмнението е пълно),  периода на орбитата може да се наблюдава директно. Ако една спектрално двойна е и затъмнителна, то колебанията на спектралните линии съвпадат по период с изменението на блясъка.\\

При спектрално двойните можем да получим крива на лъчевите скорости: функцията $v_r(t)$. Нейният вид зависи само от $e$ и $\omega$. Ако $i=90\degree$ (лъча на наблюдаване е по плоскостта на орбитата) то по $v = max(v_r)$ и така можем спектрално да определим $v$. Тогава заедно са $e$ и $\omega$ имаме всички елементи на орбитата. При $i \neq 90\degree$ получаваме линейните елементи с точност до $sin(i)$.

\subsection{Параметри на двойна система}

Движението на спътника в система, центрирана в главната звезда, се описва с Келеровите елементи:
 
\begin{itemize}
\item $a$, дължината на голямата полуос
\item $e$, ексцентрицитета на орбитата
\item $i$, ъгъла на наклон на орбитата спрямо картинната равнина
\item $P$, период на орбитата
\item $T_0$, момента на преминаване на спътника през периастъра
\item $\omega$, дължина на периастъра
\item $p$, позиционен ъгъл
\end{itemize}

\subsection{Анализ на кривите на блясъка}
Характера на затъмнението (частно, пълно или централно) се определя от $i$ и $R_1/R_2$. При $i=90 \degree$ имаме централно затъмнение. Когато дискът на едната звезда напълно закрива този на другата (пълно) в кривата на блясъка има характерни плоски да на на минимумите, защото потокът е постоянен при частични затъмнения минимумите са остри. На основа продължителността на минимумите се определя $R_1/a$ и $R_2/a$. При пълни затъмнения на основа отношението на потока в минимумите получаваме $L_1/L_2$ и ако знаем $R_1/R_2$ смятаме $T_1/T_2$. Асиметрията на момента на вторичния минимум спрямо първичните ми дава $e \cdot cos(\omega)$ (не съм сигурен дали го разчитам правилно).

%%%%%%%%%%%%%%%%%%%%%%%%%%%%%%%%%%%%%%%%%%%%%%%%%%%%%%%%%%%%%%%%%%%%
%%%%%%%%%%%%%%%%%%%%%%%%%%%%%%%%%%%%%%%%%%%%%%%%%%%%%%%%%%%%%%%%%%%%
\section{Класификация на променливите звезди}

%%%%%%%%%%%%%%%%%%%%%%%%%%%%%%%%%%%%%%%%%%%%%%%%%%%%%%%%%%%%%%%%%%%%
%%%%%%%%%%%%%%%%%%%%%%%%%%%%%%%%%%%%%%%%%%%%%%%%%%%%%%%%%%%%%%%%%%%%
\section{Еруптивни звезди}

%%%%%%%%%%%%%%%%%%%%%%%%%%%%%%%%%%%%%%%%%%%%%%%%%%%%%%%%%%%%%%%%%%%%
%%%%%%%%%%%%%%%%%%%%%%%%%%%%%%%%%%%%%%%%%%%%%%%%%%%%%%%%%%%%%%%%%%%%
\section{Звезди преди главната последователност}
\begin{itemize}
\item Хербиг Ае/Be звезди - $<10^7$ yr, $2-8 M_\odot$, свиват се и не горят H. A и B звезди с емисионни линии в спектъра
\item Хербиг-Аро обекти - мъглявини около млади звезди, образувани от джетово изхвърляне на материал който свети когато се сблъска с междузвездната материя на облака. Областите се развиват и променят в рамките на години
\item Т Таури  - еквивалент на Хербиг Ае/Be звезди, само че с $M < 2-3 M_\odot$. Освен това, докато Хербиг звезди имат изцяло лъчист вътрешен пренос, в Т Таури има и конвективна обвивка.
\item FU Orionis звезди - подлежат на бързи (години до десетилетия) промени във светимоста и спектър. Предполага се, че причината е значително акретиране на маса от обикалящия диск/материя върху звездата. Предполага се, че голяма част от звездите на ГП са минали като млади през периоди на FU Orionis активност
\item UV Ceti звезди - вид възрастни, променливи звезди, които имат силни избухвания (еруптивно променливи, flare stars). Поради това звездите стават в кратки периоди време много по-ярки (дузина пъти). Причината се смята, че са ерупции свързани с магнитни полета.
\end{itemize}



%%%%%%%%%%%%%%%%%%%%%%%%%%%%%%%%%%%%%%%%%%%%%%%%%%%%%%%%%%%%%%%%%%%%%%
%%%%%%%%%%%%%%%%%%%%%%%%%%%%%%%%%%%%%%%%%%%%%%%%%%%%%%%%%%%%%%%%
\section{Разсеяни звездни купове и звездни асоциации}
\textit{Разсеяни звездни купове и звездни асоциации. Общи характеристики и звезден състав.
Определяне на разстоянията до тях. ДХР.}

\subsection{Разсеяни звездни купове}
РЗК са звездни групировки с диаметри от порядъка на парсеци, които съдържат до няколко хиляди звезди. В МП са известни над 1100 РЗК, като се смята, че може броят им да достига 10 000. Звездите в един РЗК се формират от общ гигантски молекулярен облак и имат приблизително еднаква възраст. Те са ключови за изучаването на звездната еволюция, защото
общите им характеристики като възраст и металичност се определят по-лесно, отколкото при единични звезди. За класификация на РЗК често се използва класификацията на Тръмплер, в която първо се обозначава с римска цифра от I до IV степента на концентрация на купа (I е силно концентриран, IV е едва отличим от полето); после с арабска цифра от 1 до 3 се обозначава
доколко звездите имат сходна яркост – 1 значи, че повечето звезди имат сходен блясък, 3 – срещат се както ярки, така и слаби звезди; накрая с буква p, m или r се обозначава броят звезди в купа, от poor, medium и rich. Ако купът се намира в мъглявина, накрая се добавя n. Така Плеядите са I3rn.\\

В Галактиката разпределенията на РЗК по възраст и позиция показват някои важни особености. Ако условно наречем РЗК с възраст под $10^7$ години “млади”, тези между $10^7 - 10^9$ “средни” и тези над $10^9$ “стари”, то младите и средните РЗК са най-многобройни, докато старите са много по-малко на брой. Това разпределение отразява факта, че РЗК постепенно губят звезди за сметка на динамични ефекти – взаимодействайки една с друга, някои звезди придобиват скорости, по-високи от скоростта на откъсване и напускат купа. Взаимодействия с нееднородности в разпределението на маса в Галактиката също оказват влияние върху времето на живот на РЗК. Средното време на живот на един РЗК в Галактиката се оценява на около $3 \cdot 10^8$ години. (По-малкият брой стари купове може да е донякъде и ефект на наблюдателна селекция, т.к. в тях много от звездите с висока светимост вече са проеволюирали.)\\

Разпределението по галактични координати на РЗК показва, че младите купове заемат изключително малки z (< 0.2 kpc), т.е. те са разположени почти в равнината на диска. Средните по възраст купове достигат по-високи z (< 0.6 kpc), докато стари купове има чак до z = 2.2 kpc. Причината за това вероятно е, че на малки z приливните сили са по-големи, както е по-значимо и взаимодействието с междузвездната среда, а тези два фактора водят до разпадане на РЗК. По отношение на разпределението в проекцията на галактичната равнина, младите РЗК очертават отлично спиралните ръкави. Средните РЗК също трасират ръкавите, но по-слабо, докато старите не корелират с положението на ръкавите. Това е естествено следствие от факта, че куповете се движат в Галактиката (средната пекулярна скорост на РЗК е около 5 км/с).\\

\subsection{ДХР}
Ако нанесем на ДХР звездите от един РЗК, повечето звезди ще се намират на ГП. Най-масивните звезди ще са напреднали в еволюцията си и ще са се отделили от ГП. Мястото на ДХР, където звездите се отклоняват от ГП и поемат по клона на субгигантите към КЧГ се нарича ``точка на обръщане'' (turnoff point) и е индикатор за възрастта на купа (виж фиг. \ref{fig:hrd_clusters}). Това е така за куповете с по-голяма възраст. При младите РЗК ГП не е така добре очертана: някои звезди са “над” нея, или защото са в действителност тесни двойни системи и това покачва интегралния им блясък, или защото все още са в етап на гравитационно свиване към ГП. Особеност на ДХР е, че понякога след точката на обръщане на ГП все още има някакви звезди. Това са т.нар. “сини странници” (blue stragglers, голубые бродяги). Водеща хипотеза за обясняване на тяхното съществуване са сливания на две по-маломасивни звезди. В по-плътните части на куповете това не би трябвало да е невъзможно. Друга водеща хипотеза предполага, че това са членове на ТДС, при които има пренос на маса, забавящ еволюцията на масивната звезда.

\begin{figure}[h!]
\centering
\includegraphics[scale=0.6]{hrd_clusters.png}
\caption{Купове с различна възраст показват ``точки на обръщане'' с различно положение върху ДХР - за по-младите купове точката ще е към по-големи светимости и температури}
\label{fig:hrd_clusters}
\end{figure}

\subsection{Определянето на разстоянието до РЗК}
Разстоянието до РЗК и звезди асоциации може да стане по няколко различни начина в зависими от това колко далече са те до нас. Ако са достатъчно близки може да се приложи тригонометричния паралакс или групов паралакс.

\subsection{Тригонометричен паралакс}
Виж част \ref{sec:trig_parallax}

\subsubsection{Групов паралакс}
Виж част \ref{sec:grupov_parallax}

\subsubsection{Спектрален паралакс}
Виж част \ref{sec:spec_parallax}

\subsubsection{Стандартни свещи}
Цефеиди

\subsection{Звездни асоциации}
Звездните асоциации ЗА са млади, гравитационно слабосвързани групировки от звезди с общ произход. Броят звезди в асоциациите обичайно е между 10 и 100, а характерните им размери – десетки парсеци. Те са по-големи по размер от РЗК, но с много по-малка звездна плътност. Асоциациите се отъждествяват посредством векторите на скоростта на движението им в
Галактиката – те са приблизително еднакви. Познати са три типа ЗА:\\

1. OB-асоциации - Съдържат масивни звезди от спектрален клас О и В. Произхождат от общ обем на ГМО. О и В звездите имат силен звезден вятър, който издухва газа и праха наоколо. Това води до силно намаляване на масата в средата, благодарение на което гравитационната връзка значително отслабва. Част от формиралите се звезди в ГМО го напускат. OB-асоциациите често не са унитарни, а са съставени от отделни подгрупи звезди. ОВ-асоциациите като цяло не показват завишена звездна плътност,
ами повишена парциална плътност на звездите от този тип. Т.к. О и В звездите еволюират много бързо, ОВ-асоциациите също не живеят дълго време (няколко до десетина млн години). Исторически това е едно от първите доказателства, че звездообразуването в МП продължава и в наши дни. ОВ-асоциациите трасират спиралните ръкави.\\

2. T-асоциации - Т-асоциациите от друга страна не съдържат звезди с висока светимост. За тях са характерни обектите от типа T Tauri (може да съдържат до около хиляда такива обекти), откъдето идва и името им. Според теорията на звездната еволюция това са звезди преди ГП, откъдето следва, че Т-асоциациите също са млади формирования. Най-близката такава асоциация е Taurus-Auriga асоциацията на 140 pc от Слънцето. Често се намират в близост до молекулярния облак, от който са се формирали.\\

3. R-асоциации - Асоциации от звезди, които са потопени в отражателни мъглявини и осветяват тези мъглявини се наричат R-асоциации. Те трасират спиралните ръкави. Някои такива асоциации може би са средномасивен аналог на ОВ и Т-асоциациите, т.к. в тях се срещат Ae/Be звезди на Хербиг. Тези звезди не са достатъчно масивни, за да “издухат” междузвездната материя около себе си и затова R-асоциациите са удобни за изследване именно на междузвездната материя (например изследване на закона за екстинкцията).

\begin{figure}[h!]
\centering
\includegraphics[scale=0.8]{globular_vs_open_clusters.png}
\caption{Сравнение на разсеяни и сферичните звездни купове}
\label{fig:globular_vs_open_clusters}
\end{figure}

% \subsection{Globular cluster luminosity function}
% Има начин да се изчисли се използват и сферични звездни купове като стандартни свещи

%%%%%%%%%%%%%%%%%%%%%%%%%%%%%%%%%%%%%%%%%%%%%%%%%%%%%%%%%%%%%%%%%
%%%%%%%%%%%%%%%%%%%%%%%%%%%%%%%%%%%%%%%%%%%%%%%%%%%%%%%%%%%%%%%%%%%
\section{Сферични звездни купове}
\textit{Точки за развиване на темата: Сферични звездни купове и звездни асоциации. Общи характеристики и звезден състав.
Определяне на разстоянията до тях. ДХР.}\\

СЗК са групировки от звезди, които имат сфероподобна форма. Броят звезди в един СЗК е от порядъка на Е5 (има и с по Е6), като тези звезди обикновено са с ниска металичност (Z между -2.2 и 0) и голяма възраст – звездно население II. В СЗК не се наблюдават елементи на междузвездната среда като газ и прах, което предполага, че всичката междузвездна материя
вече е преминала през процеса на звездообразуване. Звездната плътност в СЗК е много висока, като средната ѝ стойност е около 0.4 звезди/pc 3 , а в ядрото на купа плътността може да достигне до 1000/pc 3 , т.е. разстоянието между звездите е около 1 св.г.. Повечето СЗК са разположени в халото на Галактиката. В началото на XX век именно по разстоянията до СЗК Шейпли определя размерите на Галактиката, макар и погрешно (мислел е, че RR Lyrae са цефеиди). СЗК се категоризират според концентрацията на звезди по посока центъра, като най-концентрираните са клас I, а най-разсеяните – XII. СЗК са далечни обекти и затова лесно може да се прави фотометрия на цял куп (имат малки ъглови размери). При това можем да получим интегралните им звездни величини и цветови индекси. Измерванията на интегралната абсолютна звездна величина за СЗК в Галактиката показват разпределение с мода/максимум на около -7m (като най-яркият куп, омега Кентавър, има -10m, а най-слабият, NGC 6366 има -5m) във V. Тази стойност съответства на куп с маса 2E5 Слънчеви. Началната функция на светимостите на СЗК – разпределението, което се получава при отчитане на възрастта на куповете – се оказва практически еднаква за много от близките галактики. Затова построяването на разпределението на интегралните звездни величини на СЗК в други галактики е един метод за определяне на разстоянието до тях. По аналогия за СЗК се определя и интегрален спектрален клас. СЗК имат спектрални класове в интервала F2-G4.\\

СЗК нямат идеална сферична форма. Използва се наблюдателният параметър елиптичност, или (a-b)/a, където a и b са съответно най-големият и най-малкият радиус на купа, определени чрез изофоти или пък с криви на еднаква звездна концентрация. Повечето СЗК имат елиптичност между 0.1 и 0.2. Трябва да имаме предвид, че това е наблюдателен параметър и е горна граница на истинската елиптичност на купа. Вероятно елиптичността е следствие от въртенето на купа. За СЗК са характерни пулсиращите променливи от типа RR Lyrae. Особеност на тези променливи е, че в някои купове те имат по-къси периоди, отколкото в други. Така СЗК се делят на клас I по Оостерхоф (с по-краткопериодични лириди, P\_RRab $\approx 0.55$d) и клас II по Оостерхоф (с по-дългопериодични лириди, P\_RRab $\approx 0.65$ d).\\

У няколко СЗК в Галактиката, най-ярък пример за които е омега Кентавър, се наблюдава видима дисперсия в металичността на звездите. Възможно е дори да се наблюдават три подсистеми звезди с различни металичности. Този наблюдателен факт се обяснява добре с хипотезата, че такива купове с голяма дисперсия на металичността в миналото са били ядра на сферични (така ли е терминът?) галактики-джуджета, където е възможно да са протичали няколко цикъла на звездообразуване, като във всеки следващ проеволюиралите звезди са изхвърляли в междузвездната среда тежки елементи и са се раждали звезди с по-
висока металичност. В ядрата на такива галактики е възможно и да са протичали сливания на няколко СЗК с различни възрасти и металичности. В повечето от обикновените СЗК обаче звездообразуването е имало само един цикъл. Слабата дисперсия на металичността при тях може да се обясни със самопоглъщане – проеволюиралите звезди изхвърлят метали, които попадат в атмосферите на другите звезди.

\subsection{ДХР}
ДХР на СЗК имат характерен за старото население на Галактиката вид. Структурата им се описва с теорията на звездната еволюция. На ДХР ГП е къса и преминава в клона на субгигантите, който пък преминава в клона на червените гиганти. След това в недрата на звездата започват хелиеви реакции и тя слиза на хоризонталния клон на гигантите, където прекарва значително време (~E9 години). Светимостта на звездите на ХКГ зависи главно от металичността им, което позволява оценка
на разстоянието до тях. Масата на хелиевото ядро слабо зависи от първоначалната маса на звездата. Затова звездите от ХКГ представляват обекти с практически еднакви по маса хелиеви ядра и обвивки с различни маси, където протича синтез на водород – слоест източник. Според масата на водородната обвивка звездата заема различен участък от ХКГ: червения или синия, които са разделени от ивицата на нестабилност, в която се наблюдават променливи от типа RR Lyrae. Цветът на ХКГ отлично корелира с типа на СЗК по Оостерхоф: куповете от тип 2 по Оостерхоф (с дългопериодични лириди) имат екстремално сини ХКГ и много ниска металичност.\\

След ХКГ звездата се движи надясно по ДХР и леко нагоре, за да се качи на асимптотичния клон на гигантите. На АКГ отиват почти всички звезди от ХКГ, с изключение на най-сините – те направо “падат” надолу и стават бели джужета. По пътя към АКГ сините ХКГ звезди преминават през ивицата на нестабилност и стават RR Lyrae или W Virginis променливи. За АКГ е характерна интензивната загуба на маса, която обогатява междузвездната среда. СЗК могат да имат разнообразни ХКГ, тъй като техният цвят зависи от металичността, но не само.\\

\subsection{Определяне на разстоянията до СЗК и възрастта им}
За определяне на разстоянието до СЗК най-често се използва средната звездна величина на лиридите в купа. Понякога се използва и положението на връхната точка на КЧГ, като се отчита и металичността; този метод се използва в извънгалактичната астрономия, преди всичко за елиптични джуджета-галактики, за които е възможно разделяне на отделните
звезди. Като цяло проблемът за определяне на разстоянието до СЗК все още съществува. Оценяването на възрастта на СЗК става най-добре по определяне на координатите на точката на обръщане. Също така се използват теоретични изохрони. Най-общо тук има два основни метода: вертикален и хоризонтален. Вертикалният се основа на разликата между звездните величини на хоризонталния клон (обикновено за такава се използва звездната величина на лиридите) и на точката на обръщане. Този метод среща трудности в определянето на звездната величина на точката на обръщане, защото там ГП завива почти право нагоре по
ординатата; също така ако в купа няма лириди, методът е неприложим. Хоризонталният метод използва разликата между цвета B-V на точката на обръщане (зависи от възрастта) и основата на КЧГ (не зависи или слабо зависи от възрастта, но зависи от металичността) – аналогичен метод се използва и при РЗК. Допреди няколко години (записките, от които чета са от 2009, така че поне до тогава) възрастта на някои купове се е оценявала на над 15 млрд години, като дори се е стигало до 19. През октомври 2013 обаче излиза публикация в ApJ с размерите на докторска дисертация (124 страници), в която се правят преоценки на възрастта на куповете и понастоящем те се вписват в резултатите на WMAP и Planck за възрастта на Вселената.
В СЗК се наблюдават и “сини странници”.\\

\subsection{Разпределение на СЗК в Галактиката}
СЗК са разпределени в халото на Галактиката, като концентрацията им най-общо намалява радиално от центъра на Галактиката. Болшинството СЗК заемат сфера с радиус ~10 kpc от центъра. Именно в тази сфера са разположени практически всички СЗК, образувани от протогалактичния облак. Те формират подсистемите на плътния диск (металичност [Fe/H] > - 1) и собственото Галактично хало (по-малко металични купове с много сини ХКГ). Малометаличните купове с екстремално червени ХКГ образуват сфероидална подсистема, наречена акреционно хало с радиус ~20 kpc от центъра. Отвъд тази сфера се намират и още
няколко купа. Акреционното хало от купове се е формирало чрез гравитационно прихващане на галактики-спътници.\\

Разпределението на СЗК в спиралните галактики много напомня на разпределието им в Млечния път. В Магелановите облаци обаче има различия. Там се наблюдават и сини СЗК, т.е. такива с млади звезди. Вероятно в Магелановите облаци образуването на СЗК или още продължава, или е приключило едва наскоро. СЗК еволюират динамично, губейки членове заради приливни взаимодействия с Галактиката. СЗК освен това биват деформирани заради същите взаимодействия.

%%%%%%%%%%%%%%%%%%%%%%%%%%%%%%%%%%%%%%%%%%%%%%%%%%%%%%%%%%%%%%%%%%%
%%%%%%%%%%%%%%%%%%%%%%%%%%%%%%%%%%%%%%%%%%%%%%%%%%%%%%%%%%%%%%%%%%%%%

\section{Галактиката}
\textit{Точки за развиване: Строеж на Галактиката. Спирална структура. Звездни населения и подсистеми. Начална функция на масите.}

\subsection{Строеж на Галактиката}
Галактиката в която се намираме се казва Млечния Път (МП) и е спирална галактика от тип SBbc по класификацията на Хъбъл. \textit{Начертай фиг. \ref{fig:hubble_galaxies}}. S показва, че е спирална, B, че има бар, bc показва че е между подкласите b и c. Строежът на галактиката е показан на фиг. \ref{fig:gal_parts}. \textit{Начертай строежа на МП}. МП съдържа $\approx 200 \cdot 10^9$ звезди и масата на МП е $\approx 1.5 \cdot 10^{12} M_\odot$. Болшинството звезди в нея са събрани в диска на Галактиката. Дискът има дебелина около 600 pc и диаметър около 30 kpc. Газово-праховата среда, най-младите звезди, звездните асоциации, разсеяните купове заемат относително тънък слой от диска, който се нарича плоска компонента. Именно в нея се намират спиралните ръкави. Звездната плътност силно нараства от една страна по посока Галактичната равнина и от друга – по посока Галактичния център. Именно в диаметър от около 3 kpc в центъра се намира струпване на звезди, носещо името балдж. В балджа не се наблюдават звезди, характерни за диска; там има много червени гиганти – балджът е стар. В този смисъл, дискът на Галактиката е геометрически по-скоро пръстен. В самото ядро на Галактиката се намира свръхмасивна черна дупка с маса $\approx 3\cdot 10^6 M_\odot$. Цялата тази структура е потопена в сфероидална компонента, наречена хало. Халото на Галактиката съдържа сферични звездни купове, като 90\% от тях са на разстояние под 30 kpc от центъра, а останалите 10\% достигат и до 50-60 kpc. Халото се дели на две части: вътрешната се нарича собствено хало, а външната – акреционно хало. В халото има най-голяма концентрация на тъмна материя. Галактиката се върти по посока на часовниковата стрелка (отрицателна посока), гледано от северния Галактичен полюс. Слънцето се намира на 20-25 pc северно от равнината на Галактиката и на 8 kpc радиално от ядрото (1/2 от радиуса). В Слънчевата околност средната звездна плътност е 0.1 зв/$pc^3$ , а близо до центъра на Галактиката плътността е от порядъка на 10 6 зв/$pc^3$. 

\subsection{Спирална структура}
В спиралните галактики спиралните ръкави имат повишена плътност на междузвездната среда и са области на активно звездообразуване. Ръкавите са разположени в плоската компонента. Именно в тях се намират най-младите и масивни звезди – от спектрален клас О и В, ОВ-асоциациите, младите РЗК. HII-областите също трасират спиралните ръкави: това са области от йонизиран от UV-лъчението на масивните О и В звезди междузвезден газ. Така йонизираният газ свети в оптичния диапазон. Плътните и хладни молекулярни облаци – центровете на звездообразуване – отново са обекти, намиращи се в спиралните ръкави.
Млечният път изглежда има 4 големи спирални ръкава и поне един по-малък ръкав, ръкавът Орион, където се намира и Слънцето. Спиралната структура на МП показава и въртене, което не отговаря на теоретичните предсказания основани на видимата (барионна) материя... \textit{Опиши тук феномена на тъмна материя наблюдаван чрез ротационната крива. Начертай фиг. \ref{fig:rotatsionna_kriva}.}

\subsection{Начална функция на масите}
Началната функция на масите (НФМ, initial mass function, IMF) е функция $\phi(M)$, която задава броя звезди с маси в интервала $[M, M+dM]$. НФМ изпълнява уравнението

\begin{equation}
    N|_{M_1}^{M_2} = \int_{M_1}^{M_2} \phi(M) dM
\end{equation}

като $N|_{M_1}^{M_2}$ е броят звезди с маси между $M_1$ и $M_2$. Емпирични наблюдения показват, че формата на $\phi$ е от типа

\begin{equation}
    \phi(M) \sim M^{-\alpha}
\end{equation}

сиреч $\phi(M)$ е степенна функция (power function). Стойността на $\alpha$ се променя според интервала на масите, например $\alpha = \alpha_1$ за $M \in [M_1, M_2]$, но $\alpha = \alpha_2$ за $M \in [M_2, M_3]$. НФМ е показана графично на фиг. 

\begin{figure}[h!]
\centering
\includegraphics[scale=0.25]{IMF.png}
\caption{Начална функция на масите според различни модели. Има разлики при маси $>1 M_\odot$, но за по-големи се припокриват моделите}
\label{fig:IMF}
\end{figure}

На фигурата се вижда промяната $\alpha$ според интервала на масите, но като цяло всички модели се съгласяват, че за маси $>1 M_\odot$ важи $\alpha \approx 2.35$. За маси $<1 M_\odot$ е трудно да се извлече емпирично $\alpha$, защото звезди с такива размери светят по-слабо и се намират по-трудно. Градиента с $\alpha \approx 2.35$ за маси $>1 M_\odot$ се нарича наклон на Салпетър, кръстен на ученът, който го е открил. Ако вместо броя на звездите $N|_{M_1}^{M_2}$ се търси масата на звездите в този интервал трябва да се умножи по масата преди да се интегрира:

\begin{equation}
    M_*|_{M_1}^{M_2} = \int_{M_1}^{M_2} M\phi(M) dM
\end{equation}

Често НФМ се нормализира, така че интеграла на функцията да е равен на 1:

\begin{equation}
    \int_0^{\infty} \phi(M) dM = 1
\end{equation}

Това се прави когато не е от значение абсолютната честота на звезди, ами съотношението между звездите с различни размери. И позволява функцията $\phi$ да бъде използвана за популации с различни общи маси, стига да се промени един фактор. В този вид $\phi$ представлява функция на дистрибуция на вероятноста (PDF) и $\int_{M_1}^{M_2} \phi(M) dM$ показва с каква вероятност в популацията ще попаднеш на звезда с маса в интервала $M \in [M_1, M_2]$.\\

НФМ има няколко ограничения, освен това, че малки маси е много неясна. Друго ограничение е, че НФМ е наблюдателно основана, и то на нашата околност в Галактиката. Предполагаме, че същата НФМ трябва да важи грубо и за другите части на Галактиката и дори за други галактики (стига там да има същите физични и химични качества на средата, например плътност, металичност, температура), но не можем да го докажем наблюдателно. Помощ в тази насока предлага космическия телескоп GAIA, който от няколко години предоставя много по-точни от досегашните измервания на размера и масата на голям брой звезди в МП. \textit{Тук можеш да опишеш GAIA както е представена в част \ref{sec:GAIA}}

%%%%%%%%%%%%%%%%%%%%%%%%%%%%%%%%%%%%%%%%%%%%%%%%%%%%%%%%%%%%%%%%%%%
%%%%%%%%%%%%%%%%%%%%%%%%%%%%%%%%%%%%%%%%%%%%%%%%%%%%%%%%%%%%%%%%%%
\section{Газови и прахови мъглявини}
\textit{Tочки за развиване: газови и прахови мъглявини. Механизми на светене. Дифузни и кометарни мъглявини. Области на йонизиран водород. Планетарни мъглявини.}\\

Междузвездният газ е основната компонента на междузвездната среда (МС). 90 \%  от масата на МС е съсредоточена е съсредоточена в неутралния водород HI. Той е разпространен относително хомогенно в диска и част от халото. Наблюдава се в радио линията на 21 см и в отделни абсорбции в UV. Междузвездният прах е добрe смесен с газа. По маса в галактиката той е 1 \% от МС. Праховите частици са с типичен размер 1 $\mu m$. По-голямата част от междузвездния прах произхожда от развити звезди на етапа RGB/AGB. Силните вътрешни конвекции на такива звезди отвеждат тежки елементи като C, O, Si към външните слоеве на гиганта, където вследствие изтичане на маса те навлизат в МС.\\

Типичните размери на облаците от атомарен газ HI са $R \approx 5 pc$, а масата $M \approx 500 M_\odot$. Те не са гравитационно свързани вътрешно и са в равновесие само ако външното налягане го създава.\\

Областите HI на молекулен водород, $H_2$, още наричана и молекулярни облаци, са студени: $T < \approx 3-20$ K. $H_2$. Mолекулата $H_2$ практически не излъчва, това се дължи на факта че е симетрична. По-точно, при такива ниски температури молекулите не подлежат на енергийни промени в електронни и вибрационни нива, а само на ротационни. Но $H_2$ молекулата, и някои други молекули, не разполагат с електрически дипол момент, който е нужен за наличие/преминаване в различни енергийни нива. Заради това при толкова ниски температури/енергия $H_2$ (практически) не свети\footnote{Прочети пове тук: \url{https://en.wikipedia.org/wiki/Rotational_spectroscopy\#Microwave_and_far-infrared_spectra}}. Тези облаци се наблюдават в радиодиапазона, защото в тях е добре размесена молекулата $CO$, която има електрически дипол момент и излъчва радиоемисии. Малките, тъмни облаци имат $M = 10^2 - 10^4 M_\odot$ и $R = 1-10$ pc, докато гигантските молекулярни облаци (ГМО) - $M = 10^5 - 10^6 M_\odot$ и $R = 10 - 60$ pc.  Те са вътрешно гравитационно свързани и са области на звездообразуване. Праховите частици близо до горещи OB- звезди са изложени на UV-лъчение. Те го поглъщат и излъчват в MIR. Този процес е важен за охлаждането на области на звездообразуване.

\subsection{Дифузни мъглявини}

Дифузните мъглявини са ярки мъглявини, които биват три типа: емисионни, отражателни и остатъци от свръхнови.\\

Емисионните са облаци от йонизиран газ, така наречени HII области. Облак от атомарен HI газ бива йонизиран от UV лъчението на близка OB- звезда. Следва рекомбинация до възбудено състояние, а после преход към основно в оптиката. HII - областите излъчват в балмерова серия, oсобено $H\alpha$ \footnote{$H\alpha$ е лъчение излъчено при енергиен преход във водорода от n=3 $\rightarrow$ n=2. Всички преходи към n=2 са част от Балмеровата серия, към n=1 - от Лимановата, към n=3 - от Пашеновата}. Също в спектъра се наблюдават OIII, NIII, [OIII]. HII - областите се намират в спиралните ръкави в спиралните галактики. В елиптичните галактики рядко се срещат, защото последните се образуват при сливане на галактики, при което се дава тласък на звездообразуването и МС обеднява.\\

Отражателните мъглявини са прахови облаци с характерен синкав цвят. Те отразяват Светлина от близки звезди, като поради размерите си праховите частици отразяват повече синия край на спектъра. Тъй като праховите частици се ориентират по галактичното магнитно поле, спектъра на тези мъглявини е слабо поляризиран. Отражателните мъглявини са сходни с абсорбционните (също наричани тъмни) мъглявини, но са наблюдавани от различна геометрия. Тъмни мъглявини са такива, които не позволяват на наблюдателя да види звездите зад тях. Все пак такива мъглявини могат да светят в далечният IR и в радиото, като излъчват енергията, която са погълнали в по-къси дължини.\\

Остатъците от свръхнови (super novae remnants, SNR) се развиват в три основни етапа. Първо, след избухването ($\approx 100$ yr) ударният фронт се разширява с $v \approx const$ грубо докато масата на отместеното междузвездно вещество достигне тази на изхвърлената обвивка. Това става при размер $R \approx 1$ pc. После фронтът се забавя и затова средата се нагрява до $T \approx 10^7$ K и се охлажда чрез спирачно лъчение, но с малко сечение (неефективно). Това трае $t \approx 10^4$ yr. След това при достигане на $T \approx 10^6$ K може да настъпи рекомбинация на електрони с метали и обвивката ``светва'' по-ярко, при което се охлажда все по-бързо. Скоростта намалява и може да стане с обратен знак. След $t \approx 10^6$ yr SNR вече е смесена с междузвездната среда. Лъчението и тогава покрива целия Спектър.\\

Планетарните мъглявини се образуват в последните етапи от развитието на мало и средно масивни звезди (виж частта за звездно развитие). Ниските плътности позволяват наличието на забранени линии в спектъра, например [OIII], [OII], [NII].\\

\subsection{Kометарни мъглявини}
Някои отражателни мъглявини имат кометарна форма и се наричат кометарни. В `` главата'' на `` кометата''/ мъглявината се намира TTau - звезда, която осветява мъглявината. Спектърът на тези мъглявини показва променливост, свързана с тази на TTau- Звездата.


%%%%%%%%%%%%%%%%%%%%%%%%%%%%%%%%%%%%%%%%%%%%%%%%%%%%%%%%%%%%%%%%%%%%
%%%%%%%%%%%%%%%%%%%%%%%%%%%%%%%%%%%%%%%%%%%%%%%%%%%%%%%%%%%%%%%%%%%%
\section{Разстояние до звездите}
\textit{Точки за развитие на темата: Разстояние до звездите. Тригонометрични, средни, групови и спектрални паралакс и на звездите. Собствени движения. Движение на слънцето в галактиката. Въртене на галактиката.}

\subsection{Тригонометричен паралакс}
\label{sec:trig_parallax}
Базира се на годишното отместване на звездите върху небесната сфера. Причина за което е орбиталното движение на земята около слънцето. На фиг. \ref{fig:trigon_parallax} земната Орбита е приета за окръжност ($\epsilon =0$).  Фигурата показва близка звезда на разстояние $R$ от земята. Тъй като $R_e>>a$ ще считаме, че $R_{* \oplus} \approx R_{* \odot}=R$. *' и *''  са проекциите на Звездата върху небесната сфера на фона на по далечните звезди. Ъгълът $2\pi$  е паралактичното отместване, а $\pi$ се нарича паралакс на Звездата. Паралаксът $\pi$  ъгълът, от който от звездата се вижда радиусът на земната Орбита.  За малки ъгли $\pi$  важи:

\begin{gather}
    sin(\pi) = \frac{a}{R}\\
    \pi << 1 \Rightarrow \pi = \frac{a}{R}
\end{gather}

за астрономически цели е дефинирано разстоянието парсек. 1 парсек е разстоянието, от което 1AU се вижда под ъгъл от 1'':

\begin{equation}
    1pc := \frac{1AU}{1''}
\end{equation}

1 парсек е равен приблизително на 3.26 ly. С наземни прибори можем да измерим паралакса на звезди $R < \approx 200 pc$,  макар че дори при над 30 pc относителната грешка аз става твърде голяма.

\begin{figure}[h!]
\centering
\includegraphics[scale=0.4]{trigon_parallax.png}
\caption{Геометрия на тригонометричния паралакс}
\label{fig:trigon_parallax}
\end{figure}

\subsubsection{Телескопът Gaia}
\label{sec:GAIA}
Gaia е космически телескоп с цел точни измервания на отстоянието и скоростите на звезди в Млечния път, но и други тела в и извън МП. Изстрелян е 2013, първите данни са публикувани 2016. Освен точна астрометрия има и няколко светни канала за установяване на цвета (температурата) на телата, както и спектрален инструмент за изчисляване на червеното отместване на обектите. GAIA ще наблюдава $\approx 2 \cdot 10^9$ звезди, 1\% от всичките в галактиката. Може да измери точни разстояния на звезди далеч от нас колкото до центъра на Млечния Път. Наблюдава и пулсари, супер нови и извън галактични тела като квазари и други АГЯ. Интересни неща които е открила GAIA:
\begin{itemize}
\item hypervelocity stars със скорости над тези за напускане на МП. Интересното е, че повечето от тях навлизат в МП от някъде вън
\item 4 ръкава на галактиката
\item звезди не се създават постоянно, ами в отделни прекъснати потоци - сочи, че зведообразуването в МП е в следствие на смущения от други галактики
\item има голям поток от звезди със скорости различни от тези на околността - остатъци от сливане с друга галактика
\item по-точно изчисляване на скоростта и позицията на звезди позволява и точно определяне на големината на екзо-планети около тях. И по-точно определяне на процепа на Фулън (Fullon gap) - статистическо по-ниско присъствие на екзоланети с $R \approx 2M_\oplus$
\item по-голяма разрешителна способност за бели джуджета в ДХР показва, че те се намират главно в два режима - две успоредни прави в клона/последователността на белите джуджета
\end{itemize}

\subsection{Средни паралакси - secular parallax}
По добра точност на тригонометричния паралакс (или по-далечни разстояния на които да бъде използван) може да се постигне чрез удължаване на основата (base line) на измерванията от 2AU към по-големи отстояния. От това се възползва методът на средните паралакси (secular parallax на английски). Това става чрез използване на факта, че и слънцето се движи спрямо останалите звезди и с годините може да се наблюдава видимото изместване на звездите на небето поради този ефект. Относителното движение на слънцето спрямо другите звезди в диска на Млечния Път е $\approx 4$ AU. Този метод има обаче своите недостатъци, например върху видимото изместване на звездите има ефект и факта, че и те се движат докато слънцето се движи. За да се изчисти този шум се прилагат статистически методи. 

\subsection{Групови паралакси - moving-cluster method}
\label{sec:grupov_parallax}
Метод който преди се е практикувал и то само при купове, които са приблизително близки то Земята. Сега заменен от по-точни методи. Може да се използва при асоциации и купове, които имат обща посока на движение на своите членове. При това, движението трябва да е забележимо от земята, тоест тези обекти и да са достатъчно близко. Направленията на собствените движения на звездите в такива групировки се пресичат в така наречения радиант. Тези направления само видимо се пресичат, не действително и ефектът се дължи на двуизмерната прожекция, която ние наблюдаваме на триизмерно движение. Движението и това което виждаме ние са представени във фигура фиг. \ref{fig:cluster_meth1}.

% \begin{figure}[h!]
% \centering
% \includegraphics[scale=0.5]{cluster_meth1.png}
% \caption{Поглед към движеща се звездна асоциация/куп}
% \label{fig:cluster_meth1}
% \end{figure}

% \begin{figure}[h!]
% \centering
% \includegraphics[scale=0.2]{cluster_meth2.png}
% \caption{Геометрията при изчисляването на груповия паралкс}
% \label{fig:cluster_meth2}
% \end{figure}

\begin{figure}
\centering
\begin{subfigure}{.5\textwidth}
  \centering
  \includegraphics[width=1.\linewidth]{cluster_meth1.png}
  \caption{Поглед към движеща се звездна асоциация/куп}
  \label{fig:cluster_meth1}
\end{subfigure}%
\begin{subfigure}{.5\textwidth}
  \centering
  \includegraphics[width=.8\linewidth]{cluster_meth2.png}
  \caption{Геометрията при изчисляването на груповия паралкс}
  \label{fig:cluster_meth2}
\end{subfigure}
\caption{Скици на груповия паралакс}
\label{fig:cluster_meth}
\end{figure}

Величините които ние можем да измерим за отделна звезда в такъв куп са ъгловата скорост $d\alpha/dt$, наблюдаваният ъгъл до радианта $\theta$ и чрез червено отместване можем да измерим лъчевата скорост $v_r$ на звездата. Тези за изобразени във фиг. \ref{fig:cluster_meth2}. Чрез тези величини можем по следния начин да измерим търсеното от нас разстояние до звездата $d_{OS}$:

\begin{gather}
    \theta = \frac{d \alpha}{dt} \Delta t = \dot{\alpha} \Delta t\\
    d_{SR} = \Delta t \cdot v_t\\
    v_t : \: \: tan(\phi) = tan(90 \degree - \theta) = \frac{kateta1}{kateta2} = \frac{v_r}{v_t} \Rightarrow v_t = \frac{v_r}{tan(90 \degree - \theta)}\\
    tan(\theta) = \frac{d_{SR}}{d_{OS}}\\
    \Rightarrow d_{OS} = \frac{d_{SR}}{tan(\theta)} = \frac{\Delta t \cdot v_t}{tan(\theta)} = \frac{ \frac{\theta}{\dot{\alpha}} \cdot \frac{v_r}{tan(90 \degree - \theta)}}{tan(\theta)}
\end{gather}

Тъй като в звездната асоциация или куп има много звезди този подход на груповите паралакси може да се приложи за много звезди и от тях да се извлече средно разстояние $\bar{d_{OS}}$ до асоциацията или купа.

\subsection{Спектрални паралакси}
\label{sec:spec_parallax}
Важно е да се отбележи, че при този подход няма тригонометричен паралакс. За този подход ни е необходима Видима звездна величина $m_v$ и спектър на Звездата. По спектъра определяме спектралния клас на Звездата както и класа ѝ светимост - по-малкият радиус и по-голямата плътност (клас V) дават по-широки абсорбционни линии при една и съща температура запетая отколкото по-ниската плътност при по-големия радиус (гиганти и свръх гиганти). Така определяме положението на Звездата на ДХР. Използваме приблизителната светимост на звездата в определената част на ДХР за да сметнем абсолютната звездна величина $M_V$ чрез съотношението

\begin{equation}
    \frac{L_*}{L_\odot} = \frac{M_{v*}}{M_{v \odot}}
\end{equation}

После по закона на Погсън

\begin{gather}
    m_v - M_v = 5 log(R) - 5\\
    m_v - \frac{L_*}{L_\odot} L_{v \odot} = 5 log(R) - 5
\end{gather}

смятаме $R[pc]$.

\begin{figure}[h!]
\centering
\includegraphics[scale=0.17]{spec_parallax.png}
\caption{Скица, как от спектъра на звездата се стига до нейната приблизителна светимост, която се използва после за определяне на разстоянието до нея (спектрален паралакс)}
\label{fig:spec_parallax}
\end{figure}


\begin{figure}[h!]
\centering
\includegraphics[scale=0.45]{distance_measurements.jpg}
\caption{За завършек на тази част сравнение до какви разстояния могат да се прилагат различните похвати. Нанесъл съм сам приблизително групови паралакси и астрометрията на GAIA. Oт \url{https://sites.ualberta.ca/~pogosyan/teaching/ASTRO_122/lect23/lecture23.html}}
\label{fig:distance_measurements}
\end{figure}


\subsection{Собствени движения - proper motion}
Собствено от движение се нарича изменението на ъгловите координати на даден обект на небесната сфера, породено от относителното движение на обекта спрямо гравитационния център на Слънчевата система. Ъгловите координати са  ректасцензия $\alpha$ и деклинация $\delta$. При измерване на собствените движения не се включва годишния паралакс $\pi$.  Ако в епоха $t_1$ дадена звезда има координати $(\alpha_1, \delta_1)$ Екваториална координатна система, последваща епоха $t_2$ има $(\alpha_2, \delta_2)$, то собственото ѝ движение се опитва с величините



\begin{gather}
    \mu_\alpha = \frac{\alpha_2 - \alpha_1}{t_2 - t_1}\\
    \mu_\delta = \frac{\delta_2 - \delta_1}{t_2 - t_1}\\
    \mu^2 = \mu_\delta^2 + (\mu_\alpha cos(\delta))^2
\end{gather}

с размерност дъгови секунди за година. Естествено, $\mu \sim v_t/D$,  където $v_t$  е пространствената скорост на обекта спрямо  центъра на гравитация на Слънчевата система и проектирана върху небесната сфера, а $D$ разстоянието между обекта и  гравитационния център на Слънчевата система.\\

най-голямото собствено движение има Звездата на Бернард - $\approx 10''/yr$. Тази звезда  е в Змиеносец и е на 1.8 pc разстояние (Проксима е на 1.3).



\subsection{Движение на слънцето}

Точката на небесната сфера, към която Сочи векторът на движение на слънцето, се нарича \textit{aпекс} а нейната противоположна \textit{антиaпекс}. Ако предположим, че всички звезди освен слънцето са неподвижни, То наблюдаваните им собствени движения и лъчеви скорости ще се дължат само на движението на слънцето със скорост $v_\odot$. Произволна звезда S тогава ще има относителна скорост спрямо слънцето $-v_\odot$.  Нека $\phi$ е акълът между лъча на зрение към Звездата и слънчевия aпекс. Тогава

\begin{gather}
v_r = - v_\odot cos(\phi) \Rightarrow v_\odot = - \frac{v_r}{cos(phi)}\\
v_t = - v_\odot sin(\phi) \Rightarrow v_\odot = - \frac{v_t}{sin(phi)}
\end{gather}

Като цяло ще се създава впечатление, че всички звезди се движат към антиапекса. Тогава можем да определим $v_\odot$ чрез уравненията по-горе. Ако Разглеждаме действителните звезди със самостоятелни скорости независими от слънцето, задачата се усложнява. Обаче  за близки до Слънцето звезди самостоятелните и не скорости са случайни и заради това  изчислената усредена стойност $\bar{v_\odot}$ е доста близо до действителната скорост на слънцето. Важно е да отчетем, че далечни обекти може да имат общи (т.нар. Систематични) движения. Апекс на слънцето е в съзвездие Херкулес. Слънцето се движи натам $\approx 20$ km/s.

\subsection{Въртене на галактиката}

Когато изследваме движението на звездите в галактиката можем да построим картина на нейното въртене. Въртенето на галактиката е по часовниковата стрелка гледано от Северния галактичен полюс, намиращ се в косите на Вероника. Въртенето е диференциално като ъгловата скорост би трябвало да спада (според Кеплеровата динамика и наблюдаваната светима, и барионова материя) с разстоянието от центъра на галактиката $r$. Това може да бъде описано с уравнения по следния начин

\begin{gather}
    F_{rad} = F_g\\
    \frac{mv^2}{r} = G \frac{M(r) m}{r^2}\\
    \label{eq:v_curve}
    \Rightarrow v = \sqrt{ \frac{GM(r) }{r} }
\end{gather}

Забележете, че в уравнението гравитационно влияещата маса $M$ зависи от разстоянието $r$ от центъра на галактиката, което се дължи на факта, че масата на галактиката не е концентрирана в малък обем (както например в слънчевата система) ами е разпределена в по-голямо пространство. С други думи тяло върху тяло се намира на разстояние $рr_2=2r_1$ от центъра на МП ще има повече маса която да упражнява гравитация от колкото върху тяло което се намира на отстояние $r_1$, тъй като $M(r=2r_1)>M(r=r_1)$. Използвайки ур. \eqref{eq:v_curve} и използвайки наблюдаваната плътност на лъчистата материя (барионова) в МП може да се направи теоретично изчисление как скоростта трябва да зависи от отстоянието на центъра на МП. Тази зависимост отговаря на червената крива във фиг. \ref{fig:v_curve}

\begin{figure}[h!]
\centering
\includegraphics[scale=0.4]{v_curve.png}
\caption{Крива на скоростта $v$ на материя в Млечния Път (МП) в зависимост от разстоянието $r$ на материята от центъра на МП. В червено теоретичната крива, в синьо действително наблюдаваната.}
\label{fig:v_curve}
\end{figure}

Тя показва, че според нашите наблюдения все пак главната част на барионовата материя се намира в ядрото на галактиката и скоростта $v$ трябва да спада с разстоянието $r$. Въпреки това емпиричните наблюдения на скоростите на материята се различават значително от теорията, което е изразено от синята крива във фиг. \ref{eq:v_curve}. Всъщност се забелязва, че от определен разстояние навън скоростта става почти константна. Скоростта има върхова стойност от $\approx 250$ km/s която спада и става константа с $\approx 200$ km/s. Този ефект не се наблюдава само в МП, ами и в почти всички останали галактики. Най-наложеното обяснение на този феномен е, че има материя в галактиката, която не наблюдаваме, но има гравитационен ефект и тя дава допълнителната скорост на масата. Тази скрита маса е използвана и за обяснение на други феномени, например високите скорости на галактики в галактични купове наблюдавани още от Цвики, високата температура на между галактичен газ в същите купове, силното изкривяване на светлината от галактични купове играещи ролята на гравитационни лещи и други. Общото наименование на тази скрита материя от различни феномени е ``тъмна материя`` и нейното естество и произход са едни от главните нерешени проблеми на модерната астрофизика.

% Извори: https://en.wikipedia.org/wiki/Moving-cluster_method
% http://www.astro.ucla.edu/~wright/distance.htm
% http://www.astro.gsu.edu/~crenshaw/Moving_Clusters.pdf

%%%%%%%%%%%%%%%%%%%%%%%%%%%%%%%%%%%%%%%%%%%%%%%%%%%%%%%%%%%%%%%%%%%%%
%%%%%%%%%%%%%%%%%%%%%%%%%%%%%%%%%%%%%%%%%%%%%%%%%%%%%%%%%%%%%%%%%%%%%%%
\section{Галактики}
\textit{Точки за развитие: Основни компоненти, маси, светимости, размери, звезден състав, въртене. Класификация на галактиките. Газова и прахова компоненти на галактиките. Инфрачервено и радио излъчване}

\subsection{Основни компоненти и величини}

\subsection{Класификации на галактиките}

Хъбъл класифицира галактиките по морфологията им в така наречената камертонна класификация (виж фиг. \ref{fig:hubble_galaxies}):

\begin{figure}[h!]
\centering
\includegraphics[scale=0.4]{hubble_galaxies.png}
\caption{Класификацията на Хъбъл на галактики според тяхната морфология.}
\label{fig:hubble_galaxies}
\end{figure}

Според тази класификация съществуват три основни типа галактики: елиптични (Е), спирални (S), неправилни (IRR). Елиптичните са галактики с почти елиптични изофоти (контури за петна по които повърхностната яркост е константна). В тях с просто око не е Видима ясна структура. Те са подразделени според елиптичността (ексцентрицитет) си $\epsilon = 1- b/a$, където $b$ е късата полу-ос, $a$ е дългата.  Класата на елиптичните галактики се отбелязва като $E_n$,  при което за индекса важи $n=10 \cdot \epsilon$. $E_0$ имат кръгови изофоти, а $E_{10}$ са теоретично най елиптичните.\\

спиралните галактики имат диск със спирални ръкави и централен балдж (bulge). Те се делят на обикновени спирални S и спирални с бар SB. И двата подкласа се делят на a, b, c типове според развитието (геометрично) на спиралната структура и съотношението на светимостта на балджа или диска: тип а имат ``свити'' ръкави и ярък балдж, докато тип c  имат "развити" ръкави и блед балдж. Исторически Sa се наричат ранни,  а Sc - късни. Това няма еволюционен характер. Млечният път е SBbc.\\

Неправилните имат слаба структурна организация (IRRI) или нямат такава (IRRII). S0- галактики са преходни между елиптичните и спиралните и се наричат лещовидни. Могат да са S0 или SB0 според това дали имат бар. Те имат балдж и голяма дископодобна структура без спирални ръкави и хаотична яркост. Класификацията по морфология неизбежно зависи от наблюдателни ефекти. Например бар трудно можем да видим При галактики, които виждаме ребром. Още, има галактики, които не се вписват в тази класификация, предимно заради взаимодействия с други галактики. Освен това спектрално някои галактики силно се отличават от вариантите в които лъчението главно идва от съставляващите ги звезди. Пример за такива са активните галактики, които дължат голяма част от потока си на освобождаване на гравитационна енергия при акрецията ѝ върху свръхмасивна черна дупка. Друг пример са галактиките с активно звездообразуване (starburst Galaxies). Елиптичните галактики (почти) не показват звездообразуване, а нормалните спирални като нашата образуват млади звезди с обща маса $\approx 3 M_\odot /yr$. Starburst галактиките образуват $100 M_\odot /yr$ и повече звездна маса. Много млади звезди би трябвало да водят до силно синьо или UV лъчение, но не точно това се наблюдава. Причината за това е, че звездообразуването се случва в плътните Централни части на ГМО (?), където има много прах, който абсорбира синьото и UV лъчение. Същите галактики излъчват силно в IR и submm заради преизлъчването на този нагорещен прах.


\subsection{Елиптични галактики}
Това са широка клас галактики. Включва нормалните и гигантски елиптични, джуджета електрични (по-малки $L$ и $Z$), cD елиптичните ($M_B < \approx -25^m, \; R < \approx 1 Mpc$ (!)), които се намират само в центъра на плътни купове от галактики, Blue Compact dwarf elliptical и dwarf spheroidals ($M_B \approx 8^m$, много бледи и са наблюдавани само в местната група).

С изключение на BCD, елиптичните галактики са червени в оптиката, което предполага старо звездно население. В тях има газ и прах, но по-малко масово съотношение отколкото спиралните. Металичността им расте към центъра. За много елиптични галактики няма ясна ос на въртене. Звездите в тях се държат като частиците на идеален газ: стабилизира ги не центробежната сила, а динамичното налягане и елипсовидната форма се дължи на анизотропното разпределение на звездите по скорости. Елиптичните галактики се смята, че са възникнали при гравитационно сливане на две или повече галактики. При такова събитие дисперсиите по скорости могат да са много големи и да обяснят наблюдаваната елиптична форма.

\begin{figure}[h!]
\centering
\includegraphics[scale=0.4]{gal_properties.jpg}
\caption{от \url{https://sites.ualberta.ca/~pogosyan/teaching/ASTRO_122/lect23/lecture23.html}}
\label{fig:gal_properties}
\end{figure}

\subsection{Спирални галактики}
Спиралните галактики обхващат доста по тесен диапазон светимост има са спрямо елиптичните. При тях $L=10^9-10^12 L_\odot$, $M=10^9 - 10^{12} M_\odot$, $D=5-250 kpc$.  Близо 70 \% от СГ имат бар.  При СГ $M_{gas}/M_{galtot}=4-16 \%$, като съотношението расте към късните типове (Sc). Цветът също зависи от типа, като Sc са по-сини от Sb и от Sa (а IRR по-сини от Sc). Областите на звездообразуване в Sa са в целите ръкави, А в Sc се наблюдават ярки `` възли'' в ръкавите, където протича звездообразуване.\\

Прах (в комбинация с горещ и звезди) е основният източник на IR - поток: Sc имат повече IR - поток от Sa, а SB - повече от S. В спиралните е налице цветови градиент: червени в центъра, син и в ръкавите. Това се дължи на два фактора: металичността е по-висока в центъра, а по- металичните звезди са по-червени заради по-голямата непрозрачност: младите синкави звезди са изключително в спиралните ръкави. Спиралните ръкави съдържат Освен това HII- области, газ и прах. Те са квази стационарни вълни на плътността (там $\rho$ е са 10-20 \% повече от околната в диска). Околоосното въртене на спиралните галактики, по-точно диференциалната скорост на Материята спрямо разстоянието от центъра на галактиката, са изобразени в фиг. ... Кривата ``диск'' описва очакваното кеплерово въртене, а ``хало'' - халото от тъмна материя.

\begin{figure}[h!]
\centering
\includegraphics[scale=0.5]{spiral_v_curve.png}
\caption{Криви на скорост в зависимост от центъра на галактиката са типична спирална галактика}
\label{fig:spiral_v_curve}
\end{figure}

\subsubsection{Отношение на Тули-Фишър}
През 70те години на 20ти век Тули и Фишър наблюдават съотношение между максималната скорост на спиралните галактики (която се намира почти на плато спрямо разстоянието $р$ от центъра на галактиката дължащо се на тъмната материя, както е написано по-горе) и светимоста на галактиките. Това съотношение е намерено емпирично, като за това например помага да се наблюдават линиите на 21cm-вото лъчение на водорода - лъчението което се получава, когато електрон в основното енергийно състояние във H-атома обърне спина си. Разширението на емисионната линия на 21cm-вото лъчение показва скоростта на въртене. Емпиричната зависимост има формата на 

\begin{equation}
    L \sim v^{\alpha}
\end{equation}

като $\alpha$ е приблизително 3. Това съотношение е изобразено в във фиг. \ref{fig:tully-fischer}. Съотношението на Тули-Фишър, което позволява изчисляването на $L$, е особено ценно за изчислението на разстоянията на галактики и галактични купове за, които няма много други методи по които да се измери разстоянието. По-точно, като се изчисли $L$ и се наблюдава $m$ за галактика може чрез закона на Погсън да се изчисли и разстоянието до тази галактика.

\begin{figure}[h!]
\centering
\includegraphics[scale=0.6]{tully-fischer.png}
\caption{Тули-Фишър съотношение между ротационната скорост на спирални галактики и светимоста им. В слъчая на y-оста е нанесена не $L$, а $М$, което позволява да се сравнят разстоянията на различни купове визуално. Това е и направено с двете прекъснати линии и $\Delta D$ представлява разликата в разстоянието на наблюдателя до тях. От \url{https://www.noao.edu/staff/shoko/tf.html}}
\label{fig:tully-fischer}
\end{figure}

Теоретично съотношението може да бъде приблизително обосновано по следния начин - както писахме по-горе ротационната скорост на спирални галактики е приблизително кеплерова, с други думи за нея важи

\begin{gather}
    F_r = F_g\\
    \frac{mv^2}{r} = G \frac{Mm}{r^2}\\
    \Rightarrow v^2 = \frac{MG}{r}
\end{gather}

където $М$ е барионната маса. Също така можем да кажем, че приблизително $L \sim M$ (да не се бърка с $L \sim M^{3.5}$ за звезди на ГП). Това предположение можем да го оправдаем с това, че ако галактика има 2 пъти масата (барионна) на друга галактика, то сигурно има два пъти повече звезди, и ако има два пъти повече звезди, то би трябвало да има приблизително два пъти повече светимост. С последното получаваме

\begin{equation}
    \label{eq:L_sim_v}
    L \sim M \sim \frac{v^2 r}{G}
\end{equation}

Правим и трето предположение, че повърхностната яркост $I$ на различни галактики е приблизително константна и заради това

\begin{gather}
    I = \frac{L}{\pi r^2} = const\\
    \label{eq:r_sim_L}
    \Rightarrow r \sim \sqrt{L}
\end{gather}

Така можем да сложим ур. \eqref{eq:r_sim_L} в ур. \eqref{eq:L_sim_v} и получаваме\footnote{Цялото извличане на тази ``сметка на коляно'' може да се види тук, мин. 20:00 нататък: \url{https://www.youtube.com/watch?v=uLye5CPVH1Q}}

\begin{gather}
    L \sim \frac{v^2 \sqrt{L}}{G}\\
    \Rightarrow L \sim v^4
\end{gather}

Емпирични наблюдения показват, че съотношението $L \sim v^\alpha$ има експонент $\alpha$ по-близък до 3 от колкото 4. Разминаването на теорията и наблюденията все още се изучава и има много различни параметри на галактиката, които му влияят.

\subsection{Газови и прахови компоненти в галактиките}
Фиг. \ref{fig:gal_parts} показва различните части на елиптичните галактики на пример на Млечния Път. Спиралните галактики имат централен балдж - сферична, ярка област, която има голяма плътност от звезди, но съдържа също така газ и прах. Спиралните галактики съдържат също така доста газ и прах в спиралните ръкави. В спиралните ръкави има и области на доста студен газ и прах от порядъка на десетки К. Тези райони се виждат дори визуално като по-тъмни части на спиралните ръкави. Тези студени газови и прахови облаци са всъщност местата където се зараждат звезди, в топли облаци не се зараждат. Това се дължи на факта, че по-висока температура на газа налага по-високо налягане на газа, което противодейства на свиването на газа и зараждането на звезди. Галактичните облаци от газ и прах също така се наричат мъглявини. Те могат да се разделят в различни групи според естеството им. Едно от разделянията е в абсорбционни, емисионни, рефлективни и планетарни. Емисионните мъглявини имат силни емисионни линии, дължащи се на това, че близки звезди нагряват мъглявината (често чрез UV лъчение) и тя после излъчва чрез емисионните линии. Рефлективните мъглявини често са около зараждащи се звезди и отразяват тяхната светлина. Често тези мъглявини изглеждат по-сини, тъй като тази част от светлината по-силно се разсейва чрез Рейли разсейване от атоми и молекули (пободно на ефекта на синьото земно небе). Абсорбционните или тъмни мъглявини са много сходни с рефлективните и понякога се различават от тях само по геометрията на наблюдаване. Тези мъглявини поглъщат светлината на тела зад тях и не позволяват те да бъдат видени. Планетарните мъглявини се намират около останки на късното развитие на звезди и представляват изхвърлените външни слоеве на тези малко до средно масивни звезди. Има и свръх нова мъглявини, които са аналога на планетарните мъглявини но произлизат от по-масивни звезди и последвалите ги свръх нови.

\begin{figure}[h!]
\centering
\includegraphics[scale=0.5]{gal_parts.png}
\caption{Части на спиралните галактики на примера на Млечния Път. От \url{https://scitechdaily.com/the-merger-that-led-to-the-formation-of-the-milky-way/}}
\label{fig:gal_parts}
\end{figure}

От друга страна елиптичните галактики почти нямат газови и прахови облаци в тях. Тези облаци трябва да са били изразходени преди доста време, защото освен това се наблюдава, че звездите в елиптичните галактики са възрастни (поколение II). Поради липсата на газ и прах не се очаква да има и нови вълни на звездобразуване в елиптичните галактики - нещо, което става в спиралните. Една от теориите защо елиптичните галактики газовите и прахови облаци е, че при зараждането на елиптичните галактики е имало толкова силна вълна на звездообразуване, че тя е изразходи почти всичките налични облаци.


\subsection{Инфрачервено и радио излъчване}
Инфрачервеното излъчване на галактиките се формира от три източника: звезди, междузвезден газ и прах. Излъчването от звездите е максимално в близкия IR (едно - 3 $\mu m$). Излъчването на атомите и молекулите на междузвездната среда съставя само няколко процента от общото IR излъчване на галактиките. Главният източник на IR излъчване с $\lambda > 3 \mu m$ е топлинното излъчване на прахови частици, нагрети от лъчението на звездите. Нагряването на праха става преди всичко от UV и оптичното излъчване на близките звезди. 

\subsection{Радио излъчване на галактиките}
Радио излъчването на нормалните галактики и се формира от:

-  линиево излъчване причинено от преминаване на енергийното ниво на електрони във водорода от едно ниво на друго. Най-често става дума за 21 см-вото лъчение при промяна на посоката на спина на електрона в основното енергийно ниво на водорода. Такова лъчение идва нормално от HI райони на атомарен водород.

- топлинно лъчение - излъчване в непрекъснатия спектър, дължащо се на йонизирания водород HII. То е особено силно в газовите емисионни мъглявини около горещи и те звезди.

- синхротронно лъчение - предизвиква се от високоскоростни (релативистични) електрони, които попадайки в магнитно поле излъчват или видима светлина (ако енергията на електроните и магнитното поле са големи) или радио лъчение (ако магнитното поле е слабо). Източници на синхротрон на радио излъчване са мъглявините остатъци от избухване на СН, както и централната част на галактиката.\\

Мощността на радиоизлъчването на нормалните галактики е по-ниска от мощността на оптичното им излъчване. Галактики, при които радио светимостта е сравнима с оптичната светимост се наричат радио галактики. Излъчването на радио галактиките има не топлинен характер. Основен източник на лъчението при радио галактиките С умерена мощност е областта на ядрото - тези тези райони се наричат активни галактични ядра.

%%%%%%%%%%%%%%%%%%%%%%%%%%%%%%%%%%%%%%%%%%%%%%%%%%%%%%%%%%%%%%%%%%%%%
%%%%%%%%%%%%%%%%%%%%%%%%%%%%%%%%%%%%%%%%%%%%%%%%%%%%%%%%%%%%%%%%%%%%%
\section{Галактики с активни ядра}

\textit{Зададени точки по темата: Галактики с активни ядра. Квазари. Рентгеново излъчване. Фотометрична променливост на квазарите и активните галактики. Механизми на звездообразуване в различните типове галактики.}\\

Спектърът на нормалните галактики се дължи почти изцяло на звездите в тях, с малко участие на газа и праха. Това е топлинно излъчване, тоест излъчващото вещество е практически в термодинамично равновесие (ТДР). Звезди $T_{eff}> 40000 K$ почти няма, А тези с $T_{eff} < 3000 K$ допринасят в незначителна степен за формирането на галактичния спектър поради ниската си светимост и зависимоста на Болцман $L \sim T_{eff}^4$. Така спектърът на една нормална галактика може да се приеме за суперпозиция от планкови криви и в интервала $T \in (3000, 40000) K$, което означава $\lambda \in (4000, 20000) \mathrm{\AA}$. Ако в галактиките протича активно звездообразуване този диапазон се разширява към UV заради младите масивни звезди и към FIR заради нагретия от последните прах.

Съществуват обаче галактики, които излъчват много по-широк диапазон дължини на вълната, от радио до рентген или дори гама. Това лъчение произлиза от много малък централен район на галактиката, наречен активно галактично  ядро (AGN, АГЯ), а такива галактики се наричат активни галактики (AГ). В центъра на едно АГЯ има свръхмасивна черна дупка (СмЧД), но не всички СмЧД са АГЯ\footnote{Бележки относно разяснение: Квазарите НЕ СА галактики, които съдържат АГЯ, а квазарите са самите активни ядра. Сейфертовите галактики в този смисъл представляват малко изключение - самите галактики се казват Сейфертови галактики, а ядрата им са АГЯ на Сейфертови галактики. Части на АГЯ са СмЧД, акреционния диск, област на широките емисионни линии, област на тесни линии, тор, джет?}. АГ  биват различни типове според спектъра си, светимостта си и относителната светимост на ядрото спрямо тази на звездите. Най-общо АГ се делят на Сийфъртови галактики и квазари. Основните различия между тях са - квазарите имат по-мощен централен източник, заради това имат по-голяма светимиост, по-далечни са, предполага се, че са в елиптични галактики, докато Сийфъртовите са спирални. При Сийфортовите галактики $L_{AGN} \approx 10^{11} L_{\odot}$  (колкото на всички звезди в галактиката), а при квазарите $L_{AGN} \approx 10^{13} L_{\odot}$.  Съществуват още радио галактики, които са аналог на Сийфъртовите, но са елиптични.
 
\begin{figure}[h!]
\centering
\includegraphics[scale=1.2]{quasar_vs_seyfert.jpg}
\caption{От \url{https://slideplayer.com/slide/5101155/}}
\label{fig:quasar_vs_seyfert}
\end{figure}

\subsection{Квазари}
 Квазарите са най-ярките АГЯ. По светимост квазарите са свръхдоминаната част от галактиката. Квазарите могат да бъдат радио тихи или радио шумни според мощността на радио лъчението си. Радио шумни са само около 10 \%  от квазарите. За разлика от обикновенни галактики, които показват най-вече абсорбционни линии на звездите, галактики с квазари показват широки емисионни линии в оптиката, силно рентгеново (и гама) лъчение и фотометрична променливост с периоди от часове до години. Физическата причина за лъчение в такъв широк диапазон дължини на вълната е акреция на вещество върху свръхмасивна черна дупка (СмЧД). При това гравитационната потенциална енергия се превръща в кинетична на акретиращото вещество, а после във вътрешна енергия в акреционния диск, а най-после в лъчиста. При това светимостта може да се изрази (от $E=mc^2$) като

\begin{equation}
L=\nu \dot{m} c^2
\end{equation}

където $\dot{m}=dm/dt$  е темпът на акреция, а $\nu$ е коефициентът на ефективност (``КПД'' на черната дупка). Обръщаме внимание, че за ЧД на Шварцшилд ($v_{rot}=0$) $\nu \approx 6 \%$, а при $v_{rot}=v_{rot}^{max}$ коефициентът е $\nu \approx 29 \%$ ! Излъчването на енергия чрез акреционни дискове в АГЯ е един от енергийно най-ефективните механизми във вселената, в смисъл, че този механизъм превръща според $E=\Delta m c^2$ много порядъци повече маса в енергия от колкото става това в химически реакции на атоми и молекули или дори при ядрения синтез на водород, при последния $\Delta m = \nu \approx 0.7 \%$.\\
 
\subsection{Рентгеново излъчване}
Излъчването на АГЯ не е топлинно/ не напомня на излъчване на черно тяло. Това се дължи на факта, че към излъчването на АГЯ допринасят (релативистични) процеси от акрецията на материя. АГЯ имат силен интензитет в целия спектър, от гама до радио, включително и в рентгеновия диапазон. Рентгеновото лъчение се предполага да произлиза от обратния ефект на Комптон. По-точно топлинно лъчение от акрециония диск пътувайки през короната на диска среща силно енергетични, релативистични електрони и от взаимоотношението електроните предават енергия на фотоните и дължината им се премества поради голямото количество енергия в рентгеновия диапазон.
 
\subsection{Фотометрична променливост на квазарите}
Квазарите (и АГЯ) показват променливости в яркоста в периоди от часове до месеци. Това означава, че АГЯ трябва да са достъчно малки, че физически феномени да имат каузален ефект върху целият АГЯ. Имайки предвид, че каузални ефекти от естеството на АГЯ се придвижват с приблизително светлина скорост времето на вариаците лесно подсказва размера на самото АГЯ - ако варияцията в ярскоста е седмица, то размера на излъчващата част на АГЯ-то е $d_{AGN}=v \cdot t \approx c \cdot 7days$. Този метод за изчисляване на размера на АГЯ се нарича reverbration mapping Тъй като квазарите поглъщат огромно количество материя и количеството материя е ограничено АГЯ имат крайна дължина на ``живота''. Дължината на активността на АГЯ се смята да е от порядъка на $10^8$ години, но не зависи само от количеството газ в галактиката. Смята се се взаимодействието или сливането на галактики може да задейства активността на АГЯ. Наблюденията сочат на там, че квазарите са били много по-често срещани в ранната възраст на вселената, когато плътността на галактиките е била по-голяма и е имало повече взаимодействие между тях.

Единговата светимост или единговата граница дава възможност да се изчисли горната граница на светимост на АГЯ при която все още да има равновесие между $F_{rad}$ и $F_{grav}$ или с други думи горната граница на светимост след която газът около АГЯ почва да бъде издухан отдалечавайки се от АГЯ. Ако приемем, че $F_{rad}$ и $F_{grav}$ са сили върху един протон и приемем, че $f$ е потокът лъчение дефиниран като $\Delta E = f \Delta A \Delta t$ ($f = \frac{d^2E}{dAdt}$), $\sigma$ е сечението на Томпсън за електрон (и приемем, че електрона тегли със себе си протона) тогава стига ме до:

\begin{gather}
    F_{rad} = \frac{\Delta E}{\Delta d} = \frac{f\sigma \Delta t}{\Delta t c}\\
    L=4\pi r^2 f \Rightarrow F_{rad}=\frac{L\sigma}{4\pi r^2 c}\\
    F_{rad} = F_{grav}\\
    \Rightarrow \frac{L\sigma}{4\pi r^2 c} = \frac{GM_{AGN} m_p}{r^2}\\
    \Rightarrow L = \frac{4\pi GM_{AGN} m_p c}{\sigma}
    \label{eq:Edington}
\end{gather}

Има АГЯ които в някои периоди преминават Единговата светимост, след което тя намалява и газът обратно се връща към АГЯ. При това се наблюдават промени в яркоста на АГЯ. Ако предположим, че наблюдавана светимост $L$ от АГЯ е близка да Единговата светимост може да обърнем горното уравнение \ref{eq:Edington} така, че приблизително да изчислим максималната маса $M_{AGN}$ което действително се прилага. Но уравнение \ref{eq:Edington} важи при ограничение за предимно водороден газ и изотроптно лъчение.

\subsection{Обединен модел на АГЯ}

Според обединения модел на АГЯ разликата между различните АГ е наблюдателна. Когато праховият тор около СмЧД е по лъча на зрение и я закрива, наблюдаваме Сийфърт 2 галактика. Сейфърт 2 галактики се отличават с това, че поради геометрията на наблюдаване торът закрива областите на широки емисионни линии и остава видими по-външните и по-бавно движещи се области на тесни емисионни линии. Самият тор не е част от нито една от тези области и няма значително лъчение, сравнително. Освен това се наблюдават забранени преходи от по-малко плътният газ в тора. Сийфърт 2 галактики показват и поляризация на светлината от рефлекции на светлина от газ, ако тора закрива АГЯ. Когато торът не закрива СмЧД наблюдаваме Сийфърт 1 Галактика или квазар. Сийфърт 1 имат по-широки емисионни линии от по-горещия и по-бързо въртящ се газ близо до АГЯ. Освен това се наблюдават по-малко интензивни забранени линии, защото газът е по-гъст. При наличието на джет наблюдаваме радиолъчение заради взаимодействието на джета и магнитното поле което създават електрически заредените частици на самия този джет. Ако джетът е по лъча на зрение, наблюдаваният обект се нарича блазар. Блазарите освен това имат силно радио лъчение с плосък профил на радиото, вариация в яркоста от величината на дни, силна оптическа поляризация. Блазари, които показват плосък спектър със слаби линии се водят от клас BL Lac, по прототипа BL Lacertae. Блазарите които показват силни емисионни линии се водят клас Flat Spectrum Raddio Quasars (FSRQ). Има различни хипотези за причината на различията между BL Lac и FSRQ - нпаример FSRQ да са хронологично ранните фази на BL Lac или BL Lac да съдържат по малко газ около АГЯ. Радио галактиките са елиптични аналози на Сийфортовите галактики. Разполагат с джетове, които излъчват синхротронно лъчение и въздействат с газа, който е на пътя на джетовете. Радио галактиките биват класифицирани като Фанаров-Райли 1 или 2 (FR1, FR2) - при FR1 гaзът и джетовете отслабват с разстоянието от центъра, а при FR2 радио яркостта на джетовете и газта се повишава с разстоянието от центъра.\\

\begin{figure}[H] %h!
\centering
\includegraphics[scale=0.4]{model_agn.png}
\caption{Обединен модел на АГЯ. Скица показваща съставните части на галактика с АГЯ и различните наименования на галактики според това какво вижда наблюдателя}
\label{fig:model_agn}
\end{figure}

Относно отделните части на АГ:\\

- \textbf{СмЧД} - предполага се, че всички галактики, не само АГ, съдържат СмЧД. Активни СмЧД не са по-масивни от неактивни, но има други процеси които налагат активноста. СмЧД в центъра на Млечния Път имат маса $\approx 10^6 M_\odot$, докато АГ могат да съдържат СмЧД с маси от $> 10^9 M_\odot$. Размерът на СмЧД с $10^9 M_\odot$ (Шварцшилд радиуса) е $R_{S} \approx a_{uranus}$. Шварцшилд радиуса на черна дупка е отстоянието от центъра на дупката на което се намира хоризонта на събитията или с други думи, радиуса в който гравитационната сила е толкова силна, че дори светлина не може да я напусне - $v_{escape} = c$. Шварцшилд радиусът $R_s$ може да се изчисли лесно дори с нютонова механика:

\begin{gather}
E_{gravpot}=m\Delta \phi_g = m(\phi_{g}(r\rightarrow \infty) - \phi_{g}(r)) = m(-\frac{GM}{r\rightarrow \infty} - \frac{GM}{r}) = \frac{GMm}{r} \\
E_{gravpot}=E_{kin}\\
\frac{GMm}{r} = \frac{1}{2}m v^2\\
\mathrm{if} v=c \quad \Rightarrow  \quad r=R_s=\frac{2GM}{c^2}
\label{eq:schawrzschield}
\end{gather}

- \textbf{акреционен диск} - Ако материята падаше свободно в СмЧД, то тогава нямаше да се наблюдава излъчване на енергия. Но материята около СмЧД има момент на импулса >0, който е запазен, и заради това материята не пада свободно ами се върти около черната дупка. При въртенето частиците се трият една в друга при което се отделя топлина. Въпреки това диска няма спектър под формата на функция на Планк, защото температурата на диска не e еднаква във всичките му части, ами зависи от разстоянието от СмЧД.\\

- \textbf{област на широките емисионни линии} - това са области от газ, от които произтичат широките спектрални линии, които се наблюдават при някои АГ. Разширяването на линиите произлиза от високата скорост на обикалящият газ в тези области, а не от топлинно разширение, тъй като толкова голяма топлинно разширение би изискало много от елементите да са напълно йонизирани при което не биха били наблюдавани тези емисионни линии на преходи. Наблюденията сочат скоростите в тази област да са $\approx 10^5$ km/s. Така разширението и скоростта дават улика за лесно грубо изчисление на масата на АГЯ - тъй като тези скорости не са (или само слабо) релативистични, може да се приложи формулата \ref{eq:schawrzschield}, по-точно точката за $E_{gravpot}=E_{kin}$ и се получава\\

\begin{equation}
    M_{bh} \approx \frac{v^2 r}{2G}
\end{equation}

като за $r$ може да се използва приблизително разстоянието чрез променливостта на лъчението на АГЯ. Области с широки линии излъчват разрешени или полуразрешени емисионни линии.\\

- \textbf{области на тесни емисионни линии} - подобно на широките, но тези области имат по-ниски ротационни скорости. Това означава, че се намират по-далеч от АГЯ и поради това и са не толкова горещи. При тях се наблюдават и забранени линии на преходи, което се дължи на факта, че плътността в тези области е по-ниска.\\

- \textbf{тор} - тор от молекулярен прах, който е оптически дебел. В зависимост от геометричната ориентация този тор може да скрие областите на широки линии и заради това ориентацията е разликата между Сийферт 1 и Сийферт 2 галактики.\\

- \textbf{джет} - поток от заредени частици ускорен до релативистични или почти релативистични скорости. Джетовете са под влияние на магнитни полета, които отклоняват заредените частици по спираловидни траектории (сила на Лоренц, $\Vec{F}=q\Vec{E} + q\Vec{v} \times \Vec{B}$) и водят до синхротронно лъчение. Ако лъчат на зрение е по джета тогава наблюдаваното тяло се нарича блазар. Любопитен феномен е, че части от джетовете понякога биват наблюдавани да се движат с над светлина скорост. Тази скорост е илюзорна и се дължи на високата скорост с която се движат частите от джета и на малкият ъгъл под който се движат според лъча на зрение (фиг. \ref{fig:superlum_jet}).\\

\begin{figure}[h!]
\centering
\includegraphics[scale=0.17]{superlum_jet.png}
\caption{\textbf{t0}: The spot is at A and sends out photon1 towards the observer while the spot continues
to travel towards B with $v_s=0.9c$\\
\textbf{t1}: 100 yr have passed and photon1 is at D (AD=100ly) while the spot is at B
(AB=90yr) where it sends photon2 towards the observer\\
\textbf{t3}: photon2 arrives at the observer, about 13 yr after photon1 has arrived at the
observer. The observer measures an apparent spacial difference of 23 ly
and concludes erroneously $v=ds/dt=23ly/13yr>c$ !\\
source: youtube.com/watch?v=a\_8ZMSJnP9Q}
\label{fig:superlum_jet}
\end{figure}

\textbf{Рентгеновото излъчване} произлиза от централните области на АГЯ-акреционния диск, а праховия тор го е екранира. Рентгеновото лъчение може да се генерира и при синхротронно лъчение при радио шумните АГ. Радио лъчението в галактики с квазари е също вид синхротронно лъчение. Синхротронното лъчение се създава когато електрически заредени частици биват отклонени/ускорени от магнитно поле. АГЯ могат да изхвърлят джетове с висока скорост на частниците и когато тези джетове с йонизирани частици се отклоняват магнитни полета и излъчват синхротронно лъчение, което може да е в радио, в рентгеновия диапазон или в други част на електромагнитното лъчение. Една част от рентгеновото лъчение най-вероятно произлиза от обратният Комптонов ефект. Това е феномен при който слабоенергиен фотон си въздейства със силно енергиен електрон при което електронът прехвърлят енергия на фотона и последния увеличава енергията и честотата си. Уравнението за разсейването се описва по следния начин

\begin{equation}
    \lambda' - \lambda = \frac{h}{m_e c} (1-cos(\theta))
\end{equation}

където $\lambda'$ е дължината на вълната на фотона след разсейването, $\lambda$ преди разсейването, $\theta$ е ъгълът на отклонение на фотона. В случая на АГЯ част от фотоните които биват ускорени чрез този механизъм до рентгенови честоти са радио фотони на синхротронното лъчение, част от тях са друг вид лъчение от АГЯ-то а още една част са микровълнови фотони от фоновото лъчение.

\textbf{Фото метричната променливост} се появява в АГЯ на всякакви дължини на вълната, като обикновено на по-малки $\lambda$ периодите са по-къси, а амплитудите - по-големи. Вероятни причини за променливостта са нестабилност в темпа на акреция, магнито хидродинамични процеси или/и нестабилност в диска. Именно наличието на променливост е един от главните наблюдателни фактори в подкрепа на хипотезата за СмЧД: ако източникът варира кохерентно, то излъчващата област е свързана. Максималната скорост, с която сигналът може да се разпространи вътре в нея е $c$, тоест ако квазарите показват променливост за дни и седмици, то излъчващата област има размери светлинни дни/седмици.

\subsection{Mеханизми на звездообразуване}
\textbf{Очаква се да напишеш за звездообразуването в различни типове галактики - елиптични, спирални, неправилни.
}\\

АГ, съдържащи квазари, често са гравитационно смутени заради взаимодействия с други галактики или дори процеси на сливане. Такива приливни взаимодействия могат да позволят на газа да преодолее бариерата на ъгловия момент и да попадне в близост до ядрото. Същите процеси на приливни взаимодействия могат рязко да увеличат темпа на звездообразуване в галактиката.

От друга страна се смята, че квазарите са имали интензивни фази на звездообразуване далеч във времето, поради намереното високо количество на металични линии в спектъра им. Но се смята, че започването на активната фаза на ядрото нагрява газът в галактиката до такава степен, че се предотвратява колабирането на газови облаци и формирането на нови звезди.\\

По-общо погледнато, имайки предвид не само АГ а всички видове галактики, звездите се създават в междузвездни облаци от газ и прах. За да се създадат звездите тези облаци тябва да минат през процеси на сгъстяване и фрагментиране при които облаците обикновено се свиват/колабират в множество райони. Това за да се случи трябва да е изпълнен критерият на Джинс, който включва че хидростатичното равновесие

\begin{equation}
    dP = g\rho dR
\end{equation}

трябва да е нарушено с гравитацията доминираща над налягането. Джинс дава граница на минимални маса и радиус на район в облака, за дадени температура и плътността на газ, при които гравитацията ще надделее над налягането и района ще колабира. Тези граница могат да се извлекат от прилагането на виралната теорема и че потенциалната енергия на гравитацията трябва да доминира над кинетичната енергия на газа, която му дава налягане навън

\begin{gather}
    E_k < \frac{1}{2}E_p\\
    \frac{3}{2}NkT < \frac{1}{2}\frac{GM^2}{R}\\
    3NkT < \frac{GM^2}{R}
\end{gather}

Тук може да приложим $M=N\mu$ и $M=4/3 \pi \rho R^3$:

\begin{gather}
    \label{eq:jeans_derivation}
    3\frac{\cancel{M}}{\mu}kT < \frac{GM^{\cancel{2}}}{R}\\
    3\frac{kT}{\mu} < \frac{G}{\cancel{R}} \frac{4}{3}\pi R^{\cancel{3}} \rho\\
    \left( \frac{9}{4} \frac{kT}{\mu G\pi \rho} \right)^{1/2} < R
    \label{eq:R_jeans}
\end{gather}

като граничният радиус при който \eqref{eq:R_jeans} е действително уравнение се нарича радиус на Джинс $R_j$. По сходен начин може да се изведе и масата на Джинс $M_j$, като в неуравнение $\eqref{eq:jeans_derivation}$ се изрази не масата като функция на радиус, ами обратното, $R=(\frac{3M}{4\pi \rho})^{1/3}$:

\begin{gather}
    3\frac{kT}{\mu} < GM \left( \frac{4\pi \rho}{3M} \right)^{1/3}\\
    3\frac{kT}{\mu} \left( \frac{3}{4\pi \rho} \right)^{1/3} < GM^{2/3}\\
    \left( \frac{3kT}{\mu G} \right)^{3/2} \left( \frac{3}{4\pi \rho} \right)^{1/2} < M
\end{gather}

С тези изисквания, ако район на газов облак изпълни неуравненията и има маса или радиус по-големи от посечените, гравитацията ще надвие налягането и ще започне свиването на района.\\

Докато газът се свива той развива и въртящо движение около една ос. Това е естествено за всички космически тела които по един или друг начин минават през процес на колапс. Поради запазването на момента на импулса $\Vec{L} = m\Vec{v} \times \Vec{r}$ това води до факта, че с продължително свиване и скъсяване на $\Vec{r}$ скороста $\Vec{v}$ трябва да се увеличи, което от своя страна води до увеличаване на радиалната сила на въртенето $F=mv^2/r$ която е обратно насочена на гравитационната сила и това забавя свиването на района. Тангенциалната скорост $v$ е най-висока по екватора на въртящия се район и по отслабва с увеличение на ъгловото отстояние от екватора. Поради това части на газа, които не са на екватора колабират по бързо, като придават с времето все по ``сплескана'' по полюсите форма на района докато не се оформи диск. При продължаващото свиване на диска, при което $v$ продължава да се увеличава, ако $v$ достигне орбиталната скорост, тогава приближаването на тази част от района към центъра спира напълно. Една част от масата на облака и диска все пак акретира върху ядрото и светимостта $L$ от този процес може да се извлече по следния начин: 

\begin{gather}
    E_p = \frac{GMm}{r}\\
    L = \frac{dE_k}{dt} = \frac{dE_p}{dt} = \frac{GM dm}{rdt} = \frac{GM}{r} \dot{m}
    \label{eq:L_acretion}
\end{gather}

Горното уравнение \eqref{eq:L_acretion} важи, ако се приеме, че всичката енергия от акретирана маса се излъчва обратно и не се задържа от ядрото на свиващият се и да води до увеличение на температурата на ядрото.\\ 

Ядрото на района също се свива. За него, както и за обграждащият го газ, важи правилото, че за да може да се свива газът трябва да е прозрачен, в смисъл, че трябва да може да позволява излъчването на енергия чрез лъчение, иначе свиването не би било възможно. С напредване на времето ядрото става все по-гъсто и по някое време става непрозрачно за видимото излъчване, което противодейства на свиването. Ако ядррото е натрупало достатъчно маса свиването ще продължи и ядрото ще излъчва лъчение по механизма на Келвин-Хелмхолц. В тази фаза на развитието си ядрото представлява прото звезда. Ако ядртото продължи да трупа маса и тя доведе до достатъчно големи налягания и температури се стига до ядрен синтез на деутериум. Когато лъчението на прото звездата стане толкова силна, че то издуха останалия газ около нея и по този начин спре натрупването на маса, прото звездата преминава във фазата на звезда преди главната последователност (ГП). В тази фаза светимостта на тялото е приблизително константна, но температурата на повърхноста му се покачва. Когато налягането в центъра на тялото стане достатъчно голямо и температурата стане приблизително $10^{11}$ K звездата преди главната последователност започва да синтезира водород и с това преминава във фазата на главната последователност. С това спира и свиването на центъра на звездата, както и увеличаването на повърхностната ѝ температура. Звездата навлиза в най-дългата фаза от ядрено активния си живот.\\

Сравнявайки различните типове галактики (спирални, елиптични, неправилни) може да се отбележи, че звездообразуване в днешно време се среща в спиралните и неправилните, но не и в елиптичните. Това е така, защото спиралните галактики съдържат в ръкавите си тъмни облаци от студен, молекулярен газ, който има достатъчно ниско вътрешно налягане, че при смущения, например избухвания на свръх нови или гравитационния влияния от приближили се звезди, да може да се започне процес на свиване. Неправилните галактики също съдържат голямо количество студен прах и газ, райони на звездообразуване и отворени звездни купове. В сравнение, елиптичните съдържат много малко количество газ. Предполага се, че елиптичните се преминали през фази на силно звездно формиране, зародено от сливане на галактиката с други галактики. Знае се, че елиптичните галактики са средно по-възрастни от спиралните и се предполага, че елиптичната форма се е получила след сливането на 2 или повече галактики във времето, които са оформили елиптична. Друг аргумент в защитата на това, че елиптичните галактики може да са имали бурно минало на звездообразуване дължащо се на сливане на галактики, е че телескопа GAIA помага в изчисляването на възрастта на звездите в МП и показва, че звездите не са се зараждали непрекъснато, ами на фази съвпадащи с предполагаеми сливания на МП с други галактики. Наблюдава се, че елиптичните галактики също така имат по голямо съотношение на червения канал спрямо синия канал в сравнение с неправилни и спирални галактики - още една улика, че елиптичните галактики съдържат по-стари, червени звезди, а спиралните и неправилните съдържат по-млади и сини. Ако се групират по маса, грубо може да се каже, че обикновенно неправилните са най-маломасивни, спиралните са по-масивни и елиптичните са най-масивни, като имаме предвид, че има голяма дистрибуция на масите и припокриване на трите класи галактики.\\


%%%%%%%%%%%%%%%%%%%%%%%%%%%%%%%%%%%%%%%%%%%%%%%%%%%%%%%%
%%%%%%%%%%%%%%%%%%%%%%%%%%%%%%%%%%%%%%%%%%%%%%%%%%%%%%%
\section{Двойни и кратни галактики}

\textit{Точки за развиване: двойни и кратни галактики. Общи характеристики. Групи и купове от галактики. Крупно мащабна структура на Вселената. Проблем на Скритата маса.}
% - Общи характеристики на двойните и кратните галактики или на галактики въобще?
% - локалната група
% - морфологични класификации на галактичните купове, графика
% - тъмна материя в купове галактики (Цвики)
% - рентгеново лъчение в куповете
% - Суняев-Зелдович ефект, графика, измерване на разстояние с него, на мас, температура
% - гравитационни лещи, графика

\subsection{Общи характеристики}

 Както звездите, и галактиките често се срещат по двойки. При тях обаче е значително по-трудно да се установи дали дадена двойка е физически свързана или просто оптична. При двойните галактики движението на дадена компонента около другата става толкова бавно че наблюдателно е невъзможно да се установи. Но за сметка на това сходни червени отмествания са силни улики за това, че галактиките са не просто оптически припокриващи се.

 Освен това можем да твърдим, че две галактики образуват физически двойна система, когато се наблюдава взаимодействие помежду им. Ако между компонентите се простира ``мост'', ако се наблюдава нарушаване на морфологията им или пък ако компонентите са потопени в обща обвивка, то системата е физически двойна. Това означава, че компонентите имат общ произход, динамично обвързани са и са близки по разстояние. Трите са независими помежду си и нито едно от тях не предполага задължително друго/другите.

 Възниква въпросът, можем ли да определим масите на компонентите в една двойна система от галактики по орбиталните им характеристики? Отговорът не е съвсем ясен. Първо, можем да измерим лъчевите им скорости и по разликата им да съдим за параметрите на орбитата, но така не можем да наблюдаваме въртенето във времето, няма как да получим представа за наклона на орбитата спрямо лъча на зрение. Второ, на такива големи разстояния ($>Mpc$) можем да приемаме компонентите за материални точки, но тук има друг проблем. Някои автори показват, че много двойни галактики са потопени в общо хало от тъмна материя или пък компонентите им имат по-големи халота от единичните галактики. Тогава  отношението $M/L$ не е ясно. (горното е определено на база криви на въртене).

\subsection{Групи и купове от галактики}

 Галактиките не са равномерно разпределени в пространството. Вместо това те са организирани в групи и купове. Границата между група и купа е плавна. За група от галактики смятаме структура $N<50$ члена в апроксимиран сферичен обем с диаметър $D<2Mpc$, тоест $M \approx 10^{13} M_{\odot}$. Куп от галактики наричаме по населени и по-големи структури  $M \approx 10^{15} M_{\odot}$. Куповете от галактики са най-масивните гравитационно свързани структури в селената.\\
 
 \subsubsection{Локалната група}
 Млечният път се намира също в галактична група, която наричаме Локалната Група. В обем с $R \approx 1Mpc$ Локалната Грума съдържа близо 35 галактики. В Локалната Група има 3 спирални галактики и те са тези с най-голяма светимост - Млечният Път, Андромеда и М33. Останалите са галактики джуджета с неправилна форма. Млечният Път има 2 галактики-спътници Малкият и Големият Магеланов облак, които са и най-близките галактики до него. Наблюдения на червеното отместване показват, че Андромеда се приближа към нас и след след около 4.5 милиарда години двете галактики ще се врежат.
 
 
 \subsubsection{Проблем на скритата маса}
 Наблюденията на купове показват феномен, който при откриването е бил наречен ``проблем на скритата маса'' а сега се зачислява към тъмната материя. До откриването на проблема се стига след като се прилага теоремата на вириала, който изглежда по този начин
 
 \begin{equation}
     \big \langle E_{ktot} \big \rangle = -\frac{1}{2} \sum_{i} \big \langle \Vec{F_i} \Vec{r_i} \big \rangle = \frac{1}{2} \big \langle E_{ptot} \big \rangle
 \label{eq:virial}
 \end{equation}
 
 или с думи кинетичната енергия на излорина система в равновесие е половината от потенциалната енергия на системата. Ако се приложи тази теорема върху тестова частица (галактика) в купа получаваме
 
 \begin{gather}
     E_{k\_g} = \frac{1}{2} m_g v_g^2 \approx  \frac{1}{2} m_g 3 v_{rad} ^2
 \end{gather}
 
 където използваме, $m_g$ е масата на галактиката, $M_c$ е масата на купа, $v_{rad}$ е реалната величина която може да измерим и за общата скорост $v_g$ следва
 
 \begin{equation}
     v_g = |\Vec{v_g}| \approx  \left | vec({v_{rad}}, {v_{rad}}, {v_{rad}}) \right | = \sqrt{3 v_{rad}^2}
 \end{equation}
 
 
 За потенциалната енергия следва
 
 \begin{equation}
     E_{p\_g}=G \frac{M_c m_g}{r_g} 
 \end{equation}
 
 където $r_g$ e отстоянието на галактиката от центъра на купа. Така в крайна сметка ур. \eqref{eq:virial} ни дава
 
 \begin{gather}
     E_{k\_g}=\frac{1}{2}E_{p\_g}\\
      \frac{1}{2} m_g 3 v_{rad} ^2 \approx \frac{1}{2} G \frac{M_c m_g}{r_g}\\
      \Rightarrow M_c \approx \frac{3 r_g v_{rad}^2}{G}
 \end{gather}

Трябва да се отбележи, че това важи само за купове в динамично равновесие, с други думи системата трябва да се променя само бавно с времето и да не подлежи на външно динамично влияние. Този начин на изчисление може да се повтори върху другите галактики от купа и при представителен брой от тях се получава достатъчно точна средна стойност $\bar{M_c}$ от дистрибуцията на $M_c$ изчислени за различните галактики.\\

Проблемът на скритата маса, до който стигнал Цвики през 30-те години на 20. век, е, че при сравнение на този подход на изчисление на $M_c$ с други подходи, основаващи се на светимото вещество, се получават големи разлики в $M_c$. Например може да се изчисли масата на средно статистична галактика от купа чрез $m_{gaver} = \overline{M_c}/N$, където $N$ е броят наблюдавани галактики, и да се сравни $m_{gaver}$ с типичната или очакваната маса на галактика от този вид и морфология. Може да се използва и общата светимост на купа и да се провери отношението $\overline{M_c}/L_c$ дали е в очакван диапазон. И двата подхода на лъчистата маса показват, че всъщност виралната маса е много повече от колкото светимоста на галактиките и куповете би подсказала, ако приемем че светимоста идва главно от звезди. Обикновено в куповете става въпрос за фактор на несъотношение от порядъка на $m_{viral}/m_{lum} > \approx 10$. С напредъка на измерванията на космически маси и маси на други обекти са показали нуждата от много по-високи стойности от лъчисто (барионно) наблюдаваните и проблем е получил общото наименования на проблем на \textit{тъмната материя}.


\subsubsection{Рентгеново лъчение от купове}
Рентгеново лъчение на купове от галактики е наблюдавано с космически телескопи като UHURU, Einstein и Rosat. Куповете са най-ярките източници на рентгеново лъчение след АГЯ, с разликата, че лъчението на куповете, не е променливо с времето и произлиза от голям обем в купа. Рентгеновото лъчение в куповете е по вид спирачно лъчение от отклонението на свободни електрони от протони в между галактичния газ (inter cluster medium, ICM). От спектъра на това непрекъснато, спирачно лъчение може да се установи температурата на излъчващия йонизиран газ, която е обикновено $\approx 10^7$ K. Този газ излъчва и в някои слаби емисионни линии, на елементи, които са силно йонизирани, например 25 кратно йонизирано желязо (1 обвързан електрон). Емпирично са намерени съотношения между масата на купа и интензитета на рентгеновото лъчение, които допълнително помагат в изчисляването на (скритата) маса на наблюдаван куп. Масата на ICM също може да бъде изчислена от лъчението, но също е прекалено малка за да обясни скритата маса на Цвики. Обикновени стойности за различните компоненти са $\approx 3\%$ маса на галактиките, $\approx 15\%$ маса на ICM, $80\%$ друга скрита маса.


\subsection{Крупно мащабна структура на Вселената}
В най-големи, космологични размери, вселената се е предполагало да се подчинява на 2 принципа - изотропия (вселената изглежда сходно във всички посоки, няма специални посоки) и хомогенност (вселената е сходна във всяко едно място). Тези два принципа са наблюдателно потвърдени в миналия век, като се наблюдава, че вселената е изпълнената с галактични купове, които се струпват по нишки/филаменти и оформят ``the cosmic web''. Струпването по такава мрежа се смята, че е предизвикано от струпване на тъмната материя. В пространството между филаментите на мрежата се намира големи пространства от празнини (cosmic voids). Както написах по-горе, в най-големи мащаби ($> 100 Mpc$) дистрибуцията на купове и празнини е равномерна във всички части на вселената и вселената се приема за хомогенна. Но относно изотропията в последните години набира популярност теорията на ``Оста на злото'', която твърди, че са наблюдавани скорости на купове, например с телескопа Планк, които скорости са ориентирани по ос в космологични мащаби и заради това изотропията е нарушена. Тази теория е оспорвана с довода, че измерванията на скоростите не са правилни, по-точно редукцията на данните не е взела правилно предвид самата скорост на наблюдателя около земята, слънцето, Млечния Път и Локалната Група.\\

Друг компонент който изпълва както галактичните купове, така и празнините е космическото микровълново фоново лъчение (CMB). То е остатък от инфлационната фаза на вселената, когато лъчението при разширението е било толкова силно червено отместено, че сега го наблюдаваме като лъчение на черното тяло с пик в микровълните отговарящ на абсолютно черно тяло с температура от $\approx 2.73$ K. Всъщност CMB е лъчението най-близкото до идеално черно тяло, което физиката познава. CMB е и един от най-силните инструменти, които използват космолозите и е лъчението използвано от телескопа Планк. То може да даде информация за това какво е било състоянието на вселената във фазата на рекомбинация, какво е количеството на тъмна енергия (източника на ускореното разширение на вселената), но също така CMB може да даде информация за самите галактични купове през които минава.\\

\subsubsection{Суняев-Зелдович ефект - СЗЕ}
\label{sec:SZE}
Един от методите за придобиване на информация относно свойствата на галактични купове е ефектът на Суняев-Зелдович (SZE, обратно Комптън разсейване). При този ефект фотони на CMB биват разсеяни чрез обратен Комптонове ефект ($E_{phot}<E_{e}$) от електрони на горещия ICM. Вероятността на такова разсейване е ниска, но въпреки това ефектът е измеруем. Нетният видим резултат от СЗЕ е, че когато CMB мине през гореща газ (в куп) се наблюдава отместване на пика на Планковата функция към по-високи енергии / отместване на средната честота на CMB фотоните към по-високи честоти.


\begin{figure}[h!] %h!
\centering
\includegraphics[scale=0.25]{SZE.png}
\caption{Ефекта на СЗЕ като отместване на спектъра на CMB към по-високи честоти. Прекъсната линия е спектъра на естествения CMB, непрекъсната е спектъра след СЗЕ разсейване}
\label{fig:SZE}
\end{figure}


СЗЕ комплементира използването на рентгеновото лъчение за извличане на плътността и температурата на ICM. Освен това СЗЕ и рентгеновото лъчение могат да се използват за измерване на разстоянието до купове независимо от червеното им изместване. За промяната на интензитета $\Delta I_f^{RJ}$ на лъчението в определена честота $f$ в Райли-Дженкинс режима на CMB като резултата на СЗЕ важи отношението 

\begin{equation}
    \frac{|\Delta I_f^{RJ}|}{I_f^{RJ}} \sim n_e L_c T_{gas}
\end{equation}

където $n_e$ е количеството електрони по лъча на наблюдение, $L_c$ е дължината на купа по лъча на зрение и $T_{gas}$ е температурата на разсейващия газ.


\subsubsection{Гравитационни лещи}
Друг ефект чрез който може да изчисли масата на купове e този на гравитационните лещи. Това е ефект на общата теория на относителността при който голяма маса (например тази на галактичен куп) изкривява лъча на светлина на друго тяло (например галактика намираща се зад купа, гледана от наблюдател) с нетен ефект изкривяване на образа на тялото, увеличение на образа или създаване на няколко образа. Ефекта на гравитационните лещи при арки е наблюдаван за първи път едва през 1980-те години. Най-често изкривените образи представляват арки с изкривяване към центъра на масата на купа. Изкривяването на образа зависи от отстоянието наблюдател - гравитационна леща - изобразено тяло, както и от масата на лещата и нейната дистрибуция. При подходяща геометрия може образа да бъде и пръстеновиден, тогава се говори за пръстен на Айнщайн. Намирането на множество образи в астрономически наблюдения на куп може да е трудно, по-точно различаването на галактиките в купа от образи на по-далечни обекти. Начините за намиране на изкривени образи измежду галактиките на купа са търсене с човешко око на сходна морфология, търсене на сходни цветове, еднакви спектри или червено отместване. Според силата на ефекта се говори за слабо гравитационно изкривяване и силно, като съм последното спадат арки, множествени изображения и пръстени на Айнщайн. Измерванията на масата на купове чрез гравитационни лещи са в съглашение с измерванията чрез виралната маса и рентгенови наблюдения и потвърждава независимо наличието на скрита маса / тъмна материя.

\begin{figure}[h!] %h!
\centering
\includegraphics[scale=0.18]{grav_lensing.png}
\caption{\textbf{a):} Овален гравитационен потенциал на галактичен куп изкривя изображение на задно тяло и създава две изображения - голяма синя и малка синя арка. \textbf{b):} Овален потенциал създава четворно изображение - четирите сини арки.\textbf{c):} Сферичен потенциал създава изображение под формата на пръстен на Айнщайн}
\label{fig:grav_lensing}
\end{figure}


\subsection{Скрита маса}
Няколко различни видове наблюдения относно галактиките не могат да се обяснят по добър начин, освен ако не допуснем, че галактиките съдържат допълнителна маса, освен тази която наблюдаваме да излъчва електромагнитно лъчение (барионна маса). Поради това, че предполагаме такъв вид маса, която не влияе на електромагнитно лъчение (или много слабо) тази маса се приема като резултат от нещо което наричаме тъмна материя. Галактични феномени, които се обясняват чрез тъмната материя са:

- ротационни криви на галактиките: Звездите в галактиките, при добра апроксимация, се движат според нютоновата механика и би трябвало с увеличени на отстоянието от центъра на галактиката да се движат по-бавно, съдейки по дистрибуцията на лъчистата барионна маса. Вместо това наблюдаваме че кривите на ротация на галактиките от едно отстояние от центъра на галактиките стават плоски или даже се покачват (константна или по-висока скорост с по-голямо отстояние). Това може да се обясни, ако предположим, че с по-голямо отстояние от центъра на галактиката всъщност има допълнителна маса, и тъй като не я наблюдаваме като барионна трябва да е тъмна материя.


\begin{figure}[h!] %h!
\centering
\includegraphics[scale=0.25]{rotatsionna_kriva.png}
\caption{Типична ротационна крива за (спирална) галактика}
\label{fig:rotatsionna_kriva}
\end{figure}


- дисперсията на скорости в елиптични галактики: подобно на спиралните галактики, в елиптичните се наблюдава дисперсия на скоростите, която изисква повече маса от наблюдаваната

- температура на ICM: от температурата на ICM може да се предложи колко трябва да е масата на галактическият куп, като се има предвид, че налягането на газа трябва да се уравновесява с гравитацията на масата на купа ($F_{press}=F_{grav}$). Наблюдаваните температури (извлечени от рентгеновото лъчение) изискват повече маса от наблюдаваната лъчиста

- дистрибуция на скоростите на галактики в галактични купове: аналогично на дисперсията на скорости на звезди в елиптични галактики по-горе, но приложено върху галактики в куп

- гравитационни лещи: ако масивен куп се намира между наблюдателя и по-далечна галактика купът може да огъне лъчите от галактиката и от това изкривяване в изображението, което вижда наблюдателя, може да се изчисли масата на купа, който играе ролята на гравитационна леща. Измерванията потвърждават, че в куповете трябва да се съдържа повече материя от колкото наблюдаваме от ICM и звезди

Кандидати за тъмна материя, които са били досега тествани или се тестват, включват бледи бели джуджета, черни дупки от ранната фаза на вселената с планкови размери, неутронни звезди и други. В ускорители за частици се търсят следи сочещи към потенциални супер симетрични частици с големи маси. Други теории за тъмната материя включват модификации на нютоновата динамика (МОНД), които предлагат корекции на гравитационните теории или предполагат, че уравненията на гравитацията се променят на големи мащаби и по този начин се опитват да обяснят наблюдавани феномени без допълнителна материя. Досега няма единодушие в научните среди за това кое предизвиква феномена на скритата маса или какво е тъмната материя.

%%%%%%%%%%%%%%%%%%%%%%%%%%%%%%%%%%%%%%%%%%%%%%%%%%
%%%%%%%%%%%%%%%%%%%%%%%%%%%%%%%%%%%%%%%%%%%%%%

\section{Разширение на вселената}
\textit{Точки за развиване: Разширение на Вселената. Закон на Хъбъл. Фридманови уравнения. Еволюция на локалните свойства на вселената (H(t), p(t)) стадия на доминиране на веществото и на стадия на доминиране на лъчението. Средна плътност и проблем на тъмното вещество и тъмната материя.}\\

\subsection{Фази на вселената}
\label{se:Fazi_na_vselenata}

\textbf{до $10^{-12}$ sec след ГВ} - Приема се, че вселената е започнала с Големия Взрив (ГВ) от сингулярност, след който, с променлив темп, се е разраствала до днешните размери. В най-ранната фаза на вселената са се зародили четирите фундаментални взаимодействия на вселената. Нехомогенноста в тази ранна фаза се смята да са дали началото на струпването на космически структури в по-късните етапи. В тази много ранна фаза за кратко време протича и \textbf{епохата на инфлация}, при която вселената се разширява рязко. След този период разширяването продължава, но с много по-бавни темпове. Смята се, че инфлацията е причината дребните нехомогенности на материята в тази фаза да не се изгладят, ами да бъдат ``надути'' до размери, в които да имат ефект върху космическото развитие.\\

\textbf{Ранен ядрен синтез и Рекомбинация, до 370 000 yr след ГВ} - Различни субатоматомни частици са образувани. Почти по равни части материя и анти материя са образувани, които анихилират един с другиго превръщайки се в електромагнитно лъчение, но началното количество на материя е малко повече от антиматерията и остава малко материя. Заформят се и първите протони и неутрони. До около 20 мин след ГВ протича ранният нуклеосинтез при който около 25\% от протоните и всички неутрони синтезират по-тежките елементи хелий и литий. Следващите $\approx 20000$ yr материята е толкова гореща, че е напълно йонизирана плазма през която светлината не може да пътува дълги разстояния без да бъде разсеяна и е вселената е била непрозрачна. До към $50000$ yr се смята че плътността на лъчението е доминирала плътността на материята, след което втората доминирана докато и двете спадат с различни темпове. Температурата продължава да спада и след около $2 \cdot 10^5$ yr след ГВ до към 370000 yr протича фазата на рекомбинация в която електроните и йоните се обвързват докато цялата материя не стане неутрално електрически заредена. В края на рекомбинацията, когато възбудените електрони спадат до основното си енергийно ниво те излъчват лъчението, което днес поради червеното му отместване наблюдаваме като микровълново фоново лъчение (cosmic microwave background, CMB).\\

\textbf{Тъмната ера и рейонизация, до $10^9$ yr след ГВ} - след рекомбинацията следва период в който материята съществува само в атомно състояние и тази материя е прекалено топла за да може под собствената си гравитация да се сгъсти. Поради това няма тогава и повечето източници на светлина, които виждаме днес - звезди, квазари, емисионни линии от газ. Поради това тази фаза се нарича ``Тъмната ера''. Въпреки това, заедно с CMB има и друг източник на лъчение от тази епоха - неутралните атоми се удрят в един другиго и в малкото останали свободни електрони, което води до обмен на прекалено ниски енергии за възбуждане дори на най-ниското $n=1$ квантово ниво във водорода, но достатъчни за обръщане на спина на електрона в H-атома. Обръщане на спина води до излъчването на лъчение с дължина 21 cm. Към 200-500 $\cdot 10^6$ yr най-ранните звезди и галактики се заформят. Звездите от това най-ранно поколение най-вероятно са били огромни и кратко съществуващи и за сега няма нито една наблюдавана звезда от това поколение. От тези звезди са се образували значителен брой от елементите налични в днешната вселена и тези елементи играят важна роля в ядреният синтез в днешните звезди както и в процеси на охлаждане на облаци чрез излъчване и др. С времето галактически купове и свръх купове се формират. С появата на първите звезди започва фазата на рейонизация, която продължава с лъчение на повече звезди и квазари до момента в който голямото мнозинство на газа във вселената е отново йонизиран и с това свършва епохата на рейонизация към 700-1000 $10^6$ yr след ГВ. Вселената днес все още има същите нива на йонизиран газа, като тогава, $\approx 98\%$ от общия.\\

\textbf{Барийонно и $\Lambda$ доминирана ера, до днес} - с напредването на времето структурите във вселената са се развивали и се формирали нови галактики и купове. Но с разширяването на вселената пада и средната плътност на барионната материя. Относително скоро, преди няколко милиарда години, плътността пада под нивото на плътността на тъмната енергия, защото последната е константна с времето. След тази точка във времето вселената продължава разширението си (в космологични размери) по ускорен темп, а не постоянен.\\

\textbf{бъдещето} - По някое време в бъдещето звездообразуването в галактиките ще затихне и спре поради изчерпване на водорода и разширяване на вселената до размери в които газовите облаци са прекалено раздалечени за да се свият. Смята се че разширяването ще продължи до там, че дори най-близките галактики до нас ще са извън видимата част на вселената. Има различни теория какви фази ще минат преди да се стигне последната фаза на вселената, но се предполага, че тя ще се характеризира с това, че вселената ще стигне до почти термодинамично равновесие/``Големия студ''/``топлинна смърт''.



\subsection{Закон на Хъбъл.}
Едвин Хъбъл пръв забелязва през 30те години на 20. век ефект днес наричан червено отместване на Хъбъл. Хъбъл наблюдавал близките около земята галактики и анализирал спектъра им. Спектрите на галактиките показвали червено отместване на емисионните и абсорбционните линии, което отговаря на това галактиките да се отдалечават от нас. Този ефект произлиза от по-общият ефект на Доплерово отместване: $v/c = \Delta \lambda / \lambda_{emit}$ за $v<<c$, а ако $v$ стане релативистична се налагат корекции. Хъбъл освен това измерил разстоянието до галактиките и забелязал, че има линейна зависимост между това колко далече е една галактика от нас, $r$, и с скоростта $v$ с която се отдалечава:

\begin{equation}
    v = H_0 \cdot r
\end{equation}

Тази зависимост се нарича \textit{закон на Хъбъл}, като $H_0$ е емпирично измерената линейна \textit{константа на Хъбъл} и има стойност $\approx 0.73+/-0.05 km/s/Mpc$. По-точно би било да се нарича параметър на Хъбъл, защото $H_0$ е константен в пространството, но не и във времето. Този закон важи за всички галактики наблюдавани от Хъбъл (и от по-късни наблюдения), без значение за положението на галактиките на небето и довел до заключението, че вселената трябва да се разширява, което значи, че галактиките трябва да се раздалечават една от друга. Това довело до края на хипотезата за статичната вселена и изисквало вселената назад във времето да е била сгъстена в малък обем от който да се разширила (Големия Взрив).

\begin{figure}[h!] %h!
\centering
\includegraphics[scale=0.55]{hubble_graph.jpg}
\caption{Оригинална фигура от един от трудовете на Хъбъл. Градиента на функцията е $H_0$}
\label{fig:hubble_graph}
\end{figure}

\subsection{Фридманови уравнения.}
Уравненията на Фридман са 2 уравнения, които обясняват как се разширява пространството на изотропичен и хомогенен космос в контекста на общата теория на относителността (ОТО). Предположение от което се нуждаят уравненията е космологическият принцип - космосът е изотропичен и хомогенен в голям мащаб, което е емпирично доказано за $>100Mpc$. Комбинирайки метриката на Робъртсън-Уокър и използвайки уравненията на Айнщайн за ОТО се извличат следните Фридманови уравнения:

\begin{equation}
    FI) \quad \frac{3(\dot{R}^2 + k)} {R^2} - \Lambda = 8\pi \rho
\end{equation}

\begin{equation}
    FII) \quad \frac{2R\ddot{R} + \dot{R}^2 + k}{R^2} - \Lambda = -8\pi p
\end{equation}

В това уравнение $k$ е параметърът на изкривяването и трябва да заеме стойност -1 (хиперболична вселена), 0 (плоска вселена) или 1 (сферична вселена). За улеснение е използвано $c=G=1$. $\Lambda$ е космологичната константа, предложена като допълнение не от Фридман, а от Айнщайн. 


\subsection{Развитие на $H(t)$ и $p(t)$ за лъчисто и материйно доминирани епохи}
От ур. на Фридман можем да направим няколко не много сложни заключения, например как се развиват $H(t)$ и $p(t)$ с времето. Например, използвайки факта, че параметъра на Хъбъл е дефиниран като $H(t):= \dot{R}/R$ ако използваме FI) можем да го пренапишем в 

\begin{gather}
    \frac{\dot{R}^2} {R^2} = H(t)^2 = \frac{8\pi \rho}{3} + \frac{\Lambda}{3} - \frac{k}{R^2}\\
    \label{eq:Hubble_par}
    H(t) = \sqrt{\frac{8\pi \rho}{3} + \frac{\Lambda}{3} - \frac{k}{R^2}}
\end{gather}

От ур. \eqref{eq:Hubble_par} можем да направим заключенията, че плътността на материята $\rho$ трябва постоянно да спада с времето. Също така и приноса $-\frac{k}{R^2}$ ще спада постоянно с времето, тъй като $R$ е ставала и ще става само по-голям като стойност, или поне това показват наблюдения и теория. Ако предположим, че се намираме в плоска вселена, което показват наблюдения, даже може напълно да отпишем този принос, защото $k=0$ в плоска вселена. Заключенията, които можем да направим за времевото развитие на $H(t)$ са тогава:

- дълъг период от време, от ГВ до края на \textbf{материйно доминираната епоха}, приносите $\frac{\Lambda}{3} - \frac{k}{R^2}$ са имали нищожно малки величини в сравнение с приноса на плътността $\rho(t)$, която обаче е спадала с времето докато вселената се е разширявала. В материйно доминираната епоха Хъбъл параметъра е имал поради това формата

\begin{equation}
    H(t) \approx \sqrt{\frac{8\pi \rho}{3}}
\end{equation}

- вселената ще стигне епоха в която все пак доминацията на $\rho$ ще бъде заместена, тъй като $\rho$ спада с разширението на вселената, и $\Lambda$ ще замести като доминираща компонента, тъй като $\Lambda(t) = \Lambda = const$. В тази \textbf{епоха на доминиране на тъмната енергия} (която всъщност е настоящата) параметърът има следната форма

\begin{equation}
    H(t) \approx \sqrt{\frac{\Lambda}{3}}
\end{equation}

Също така тази стойност представлява долната граница на параметъра към който той се стреми асимптотно и поради това можем да изобразим $H$ в зависимост от $t$ както е направено във фиг. \ref{fig:H_vs_t}

\begin{figure}[h!] %h!
\centering
\includegraphics[scale=0.45]{H_vs_t.png}
\caption{Параметър на Хъбъл срещу времето. Вижда се, че днес е доста близко до граничната стойност от $\approx 70 km/s/Mpc$}
\label{fig:H_vs_t}
\end{figure}

Друг начин да се изрази развитието на Хъбъл параметъра е чрез уравнението

\begin{equation}
    \label{eq:Hubble_par2}
    H^2 = H_0^2 \left( \frac{\Omega_r}{R^4} + \frac{\Omega_m}{R^3} + \frac{\Omega_k}{R^2} + \Omega_\Lambda \right)
\end{equation}

където $H_0$ е стойността на Хъбъл параметъра в днешно време, $\Omega_r$ е сегашната плътност на лъчението разделена на критичната плътност и за $m$, $k$ и $\Lambda$ е аналогично, отговаряйки на материята, кривината и тъмната енергия. С развитието на времето мащабният фактор $R$ нараства, което води до това всички фактори в \eqref{eq:Hubble_par2} (освен $\Omega_\Lambda$) да спадат, което води до това и $H$ да спада асимпточино до границата наложена от $\Omega_\Lambda$.


\subsection{Тъмно вещество и тъмна материя}
В края на 90те години няколко научни групи провеждат наблюдения на разширяването на вселената чрез метода на стандартните свещи. По-точно наблюдават далечни галактики за тип 1а свръх нови, тъй като поради квантови ефекти тези свръх нови имат точно определена светимост независима от отделната звезда. Константната светимост позволява измервайки видимата звездна величина да се определи сравнително лесно спрямо други начини разстоянието до свръх новите и с това до галактиките. Така става възможно работата на Хъбъл по зависимостта скорост-разстояние да бъде продължена към по-големи разстояния. Научните трудове установяват, че всъщност се наблюдава не само увеличение на скоростта с разстоянието от нас, но и ускорение на темпа на разширение. Това може да се обясни само ако се приеме корекцията показана по-горе в общата теория на относителността и Фридмановите уравнения с $\Lambda > 0$, сиреч съществуване на вид тъмна енергия, която ускорява разширението на вселената. За това откритие Рийс, Перлмутър и Шмид получават нобеловата награда за физика. По-късно измервания на плътностата на материята, $\rho$ в FI), и измервания с телескопа Планк на кривината на вселената, $k$ в FI) и FII), показват резултати в съглашенство с теорията за тъмна енергия.\\

Едно от свойствата на тъмната енергия е, че нейната средна плътност не се променя с времето - $\Lambda = \Lambda(t) = const$ - и заради това се нарича ``космологична константа''. За разлика стойностите $\rho_{m}$ и $\rho_{r}$ спадат с времето. Наблюденията показват, че с напредване на възрастта на вселената и напредване на разширението на космоса ще се стигне до $\Lambda > \rho_r$, $\Lambda > \rho_m$. Дори количеството на тъмна материя не може да промени тази тенденция и трябва да се има предвид, че докато при разширението на вселената се създава повече тъмна енергия, защото за нейната плътността важи $\Lambda(t)=const$, то същото не важи за тъмната материя която се разрежда с разширението на вселената. Всъщност ерата на $\Lambda > \rho_r$ вече е настъпила, сравнително скоро преди $3 \cdot 10^9$ yr, сиреч вече се намираме в ерата на вселената доминирана от тъмна енергия. Приема се, че поради това разширението на вселената няма никога да бъде спряно и ще продължи с ускорени темпове и крайната ера на вселената ще бъде тази на ``топлинната смърт на вселената'' при която вселената ще стигне състояние (много близко до) максимална ентропия. Друга възможност е преди да се стигне до тази фаза да се стигне до ``големия разрив'', епоха в която разширението даже ще доминира над елементарните сили на атомите и ще раздалечи дори елементарни частици на безкрайни разстояния едно от друго.


\begin{figure}[h!] %h!
\centering
\includegraphics[scale=0.55]{densities_universe.jpg}
\caption{плътностата на различните съдържания на вселената като функция на времето. Показва кога коя плътност доминира - до $10^5$ yr има лъчиста доминация, после до $10^{10}$ yr доминира материята и от тогава до $\infty$ доминира тъмната енергия}
\label{fig:densities_universe}
\end{figure}


\begin{figure}[h!] %h!
\centering
\includegraphics[scale=0.85]{skitsa_expans_universe.pdf}
\caption{Скица показваща как SNA1 трябва да изглеждат разпределени в d-v диаграма и как да се интерпретира темпа на разширението от такава диаграма}
\label{fig:skitsa_expans_universe}
\end{figure}


\begin{figure}[h!] %h!
\centering
\includegraphics[scale=0.65]{adam_riess_sn1a_obs.jpg}
\caption{Резултатите от наблюденията на Адам и Рийс на SN1A довели до Нобеловата награда за двамата}
\label{fig:adam_riess_sn1a_obs}
\end{figure}


\subsection{Вакумна катастрофа или проблем на 120-те порядъка}
Едно от предположенията за произхода на тъмната енергия е, че тя е всъщност енергията на основното състояние $E_0$ на скаларно поле от теорията на полетата на квантовата механика (quantum field theory, QFT). Основното състояние на такова поле трябва да е доста сходно до основното състояние на квантово-механичен осцилатор. За енергията на основното състояние на последния важи:

\begin{equation}
    E_0 = \sum_i \frac{1}{2} \hbar \omega_i
\end{equation}

В това уравнение енергията е сбор от приноса на всички модове на разрешено вълнение. Същото не може да важи за квантово механично поле, защото сбора на безкрайно много модове би бил безкрайна стойност. Заради това предложеното решение е да се наложи максимална честота $\omega_{max}$ над която се предполага, че няма вече приноси към енергията, защото там квантовата механика ще се сблъска с проблеми на гравитацията. Тази граница се приема да е при енергията на Планк. Така може грубо да пресметнем, че ако предположим, че изчисляваме осцилации в достатъчно голям обем сумата $E_0$ ще стане на интеграл в три измерения в който интегрираме честота и крайният резултат е пропорционален на $\omega^4$, по-точно на граничната честота на интегрирането:

\begin{equation}
    E_0 \sim \rho_\Lambda \sim \omega_{pl}^4 \sim \frac{1}{\lambda_{pl}^4}
\label{eq:e_vac}
\end{equation}

При енергията на Планк се приема, че важи, че комптоновата дължина на вълната е равна на Шварцшилд радиуса му. Комптоновата дължина на частица, е дължината на вълната на хипотетичен фотон, който енергийно би отговарял ($E=hf$) на масовата енергия на тази реална частица ($E=mc^2$). Това може да се изрази обърнато според дължината на вълната като:

\begin{equation}
    E_c = hf = \frac{hc}{\lambda_c} = mc^2 \Rightarrow \lambda = \frac{h}{mc}
\end{equation}

За радиуса на Шварцшилд важи $v_{esc}=c$ или по-точно

\begin{gather}
    E_{pot} = E_{kin}\\
    G\frac{m_1 m_2}{r_s} = \frac{1}{2}m_1 v^2\\
    \Rightarrow m_2=m_{pl}= \frac{r_s v^2}{2G} = \frac{\lambda c^2}{2G}
\end{gather}

Горните 2 уравнения може да се обединят в

\begin{gather}
    \lambda_{pl} = \frac{h}{m_{pl}c} =  \frac{h}{c} \cdot \frac{2G}{\lambda_{pl} c^2}\\
    \lambda_{pl} = \sqrt{\frac{2Gh}{c^3}}
\end{gather}

Този резултат може да се комбинира обратно в ур. \eqref{eq:e_vac} за да се изчисли много грубо предполагаемата вакумна енергия и да се сравни с енергията нужна за ускорението разширение на вселената, което се забелязва. Предполага се, че може да има разлики от теоретичната стойност и наблюденията от $\approx 10^1$, но разликата в енергията всъщност е далеч по-голяма - теоретичната стойност надхвърля наблюдаваната с $\approx 120$ порядъка. Това се нарича вакумната катастрофа или проблема на 120-те величини.


%%%%%%%%%%%%%%%%%%%%%%%%%%%%%%%%%%%%%%%%%%%%%%%%%%%%%
%%%%%%%%%%%%%%%%%%%%%%%%%%%%%%%%%%%%%%%%%%%%%%%%%%%%%%%

\section{Еволюция на ранната вселена}
\textit{Точки по които да се развива: Основни процеси в еволюцията на ранната вселена. Космично микровълново лъчение - характеристики, анизотропия, произход. Космологичен нуклеосинтез. Ограничения, следващи от космологичния нуклеосинтез.}

\subsection{Основни процеси в еволюцията на ранната вселена}
\textit{Същото като в част \ref{se:Fazi_na_vselenata}}. 



\subsection{Космично микровълново лъчение}

Космичното микровълново лъчение (КМЛ, също наричано микровълново фоново лъчение, CMB) е лъчение което е равномерно лъчение в цялата вселена и е остатък от ерата на рекомбинацията от $\approx 380 000$ yr след ГВ. Тогава свободните електрони са се свързали с наличните атомни ядра и са създали неутрално заредени атоми. При този процес, и при процеса на последващата релаксация на електроните от възбудено състояния към основното енергийно състояние на атомите е излъчено това лъчение. Лъчението е най-близкото лъчение до черно тяло която емпирично е наблюдавано, като нехомогенностите в лъчението са от порядъка на $1/10^5$. Планковата функция на КМЛ има пик в микровълновият режим отговарящ на температура от 2.73 К. Всъщност, когато е било излъчено КМЛ е имало много по-висока честота на пика отговаряща на черно тяло с 3000 К, но поради разширението на вселената лъчението е червено отместено. Предполага се, че анизотропичностите в КМЛ всъщност са се зародили през епохата на космическата инфлация, когато вселената е била на възраст $<10^{-30}$ sec. По време на инфлацията тези различия в плътността на материята са “раздути” до космологични размери и са се отразили в струпването на структури по-късно, както и в КМЛ.\\

\begin{figure}[h!] %h!
\centering
\includegraphics[scale=0.55]{universe_curv.png}
\caption{Възможните форми на вселената според възможните плътности от ур. на Фридман и съпътващата ги стойност $k$ на кривината от същите ур. От \url{https://www.astronomynotes.com/cosmolgy/s9.htm}}
\label{fig:universe_curv}
\end{figure}

Дребните анизотропии в КМЛ, с амплитуди от $\approx 10^{-5}-10^{-4}$ K, са наблюдавани с космическите телескопи WMAP и Планк. Тези анизотропии помагат да се извлекат качества на вселената и на ранния ѝ период на развитие. Например заключението, че съдържанието на енергия и маса във вселената е $\approx 75\%$ в тъмната енергия се дължи на измервания на анизотропиите с Планк. Друг пример е измерването на кривината на вселената $k$ от уравненията на Фридман. Това $k$ отговаря на това дали вселената е отворена (сбора от ъглите в триъгълник е по-малък от 180$\degree$, паралелни лъчи се раздалечават на космологични разстояния), затворена (>180$\degree$) или плоска (=180$\degree$) (виж фиг. \ref{fig:universe_curv}). Тригонометрични измервания на огромни разстояния чрез анизотропиите показват, че вселената е почти съвършено плоска. Друо нещо което може да се наблюдава в КМЛ са baryon acoustic oscilations (BAO). Те са огромни звукови вълнения, които са се разпространявали във вселената по време на фазата преди рекомбинацията. BAO са се създавали от осцилации от свиване и разширяване на газа по време на рекомбинацията под влиянието на гравитацията и вътрешното налягане на газа. С настъпването на рекомбинацията тези осцилации, по-точно техните суперпозиции, са отпечатани в КМЛ и даже се наблюдава дистрибуцията на галактики да следва суперпозицията на райони с по-голяма плътност произлизащи от BAO.\\


\begin{figure}[h!] %h!
\centering
\includegraphics[scale=0.30]{BAO.png}
\caption{Скица на суперпозицията на BAO и струпването на галактики около тях}
\label{fig:BAO}
\end{figure}

Начинът по който количествено се измерва КМЛ е чрез фитиране на спектър на мощността (power spectrum), който може да се интерпретира като хистограм на различните модове на осцилациите, виж фиг. \ref{fig:cmb_power_spectrum}. Модовете на x-оста отговарят на това колко пъти се е свил и разширил районът на сгъстяване. На y-оста е сложена температурата на флуктуаците, което отговаря на това колко често се срещат такива модове. Най-високият пик при $1 \degree$ отговаря на осцилации, които от създаването си са имали възможността само веднъж да се свият във времето от големия взрив до рекомбинация. Втория пик отговаря на осцилиращи райони които са се свили и са разширили веднъж. Третия пик на райони, които са се свили, разширили, и после свили втори път. И.т.н. със следващите пикове. Различни числени модели са правени, които съдържат различни стойности на космичните параметри и дават различни изкуствени спектри на мощността. Тези изкуствени спектри се сравняват с истинския измерен спектър от CMB и се търси най-доброто съвпадение. Един от числените похвати, които може да се приложи в такъв случай е Markov chain Monte Carlo (MCMC), при който се търси минимума на функция с много параметри (в много измерения) и минимума отговаря минималната разлика на модела с действително наблюдавания спектър. Така могат да се определят стойностите на космическите параметри (фиг. \ref{fig:cmb_models}).

Едно от нещата които могат да се тестват с BAO е кривината на вселената $k$ от Фридмановите уравнения. За целта се измерва средния ъглов размер на най-често срещаният мод на BAO. Този мод отговаря на райони, които са имали възможността само веднъж да се свият за времето на плазмата в която се намират, сиреч за 380 000 год. От това може да извадим и теоретична стойност за размера $D$ на този мод с отношението

\begin{equation}
    D \sim  v_s t f_{exp}
\label{eq:BAO1}
\end{equation}

като $v_s$ е скоростта на звука в плазмата, $t$ е времето за разпространение на вълната на мода ($\approx 380000$ yr) а $f_{exp}$ е фактор за компенсацията от разширението на вселената. Ако приемем, или решим да тестваме, Евклидова метрика, която важи в плоска вселена с $k=0$, следва за ъгловата големина $\theta_D$ на този мод чрез тригонометрия

\begin{equation}
    tan(\theta_D) = \frac{kateta1}{kateta2} = \frac{D}{d_D}
\label{eq:BAO2}
\end{equation}

където $d_D$ е разстоянието от наблюдателя до KML. Най-накрая като сравним наблюдателният размер на този най-чест мод (според фиг. \ref{fig:cmb_power_spectrum} е $\approx 1 \degree$) и като сравним теоретично изчислената стойност на $\theta_D$ от ур. \eqref{eq:BAO1} и \eqref{eq:BAO2} можем да кажем дали предположението за Евклидова метрика (лъчи остават паралелни, сбор на ъглите на триъгълник е $180 \degree$) е верен за космологичните размери на вселената и с това кривината на вселената е $k=0$. И действително тези сравнения ни дават резултат, че в рамките на точността на измерванията вселената е много плоска/Евклидова. Също така, това измерване на $k$ ни показва според Фридмановите уравнения общото количество на енергия от $\Lambda$, $\rho$ и $p$:


\begin{gather}
    FI) \quad \frac{3(\dot{R}^2 + k)} {R^2} - \Lambda = 8\pi \rho\\
    FII) \quad \frac{2R\ddot{R} + \dot{R}^2 + k}{R^2} - \Lambda = -8\pi p
\end{gather}


\begin{figure}[h!] %h!
\centering
\includegraphics[scale=0.45]{cmb_power_spectrum.png}
\caption{Дистрибуция/честота на различните модове на осцилации в KML}
\label{fig:cmb_power_spectrum}
\end{figure}



\begin{figure}
\centering
\begin{subfigure}{.5\textwidth}
  \centering
  \includegraphics[width=.95\linewidth]{cmb_power_spec1.png}
%   \caption{A subfigure}
  \label{fig:sub1}
\end{subfigure}%
\begin{subfigure}{.5\textwidth}
  \centering
  \includegraphics[width=.95\linewidth]{cmb_power_spec2.png}
%   \caption{A subfigure}
  \label{fig:sub2}
\end{subfigure}
\caption{Два различни модела за модовете на BAO. Моделите се различават в плътността на барионна материя $\Omega_b$, показана като розова колона. Тази разлика дава различен спектър на мощността. Фитирането на такъв модел върху реално наблюдаваните амплитуди на модовете показва какво е количеството на $\Omega_b$ във вселената. Сходни модели и фитове се правят за изчисляването на останалите компоненти от енергията - тъмната материя и тъмната енергия.}
\label{fig:cmb_models}
\end{figure}


\subsubsection{Суняев-Зелдович ефект - СЗЕ} 
\textbf{Същото като в част \ref{sec:SZE}}. Един от методите за придобиване на информация относно свойствата на галактични купове е ефектът на Суняев-Зелдович (SZE, обратно Комптън разсейване). При този ефект фотони на CMB биват разсеяни чрез обратен Комптонов ефект ($E_{phot}<E_{e}$) от електрони на горещия ICM. Вероятността на такова разсейване е ниска, но въпреки това ефектът е измеруем. Нетният видим резултат от СЗЕ е, че когато CMB мине през горещ газ (в куп) се наблюдава отместване на пика на Планковата функция към по-високи енергии / отместване на средната честота на CMB фотоните към по-високи честоти.


\begin{figure}[h!] %h!
\centering
\includegraphics[scale=0.25]{SZE.png}
\caption{Ефекта на СЗЕ като отместване на спектъра на CMB към по-високи честоти. Прекъсната линия е спектъра на естествения CMB, непрекъсната е спектъра след СЗЕ разсейване. Силата на ефекта е силно преувеличена}
\label{fig:SZE}
\end{figure}


СЗЕ комплементира използването на рентгеновото лъчение за извличане на плътността и температурата на ICM, както и техните градиенти. Освен това СЗЕ и рентгеновото лъчение могат да се използват за измерване на разстоянието до купове независимо от червеното им изместване. За промяната на интензитета $\Delta I_f^{RJ}$ на лъчението в определена честота $f$ в Райли-Дженкинс режима на CMB като резултата на СЗЕ важи отношението 

\begin{equation}
    \frac{|\Delta I_f^{RJ}|}{I_f^{RJ}} \sim n_e L_c T_{gas}
\end{equation}

където $n_e$ е количеството електрони по лъча на наблюдение, $L_c$ е дължината на купа по лъча на зрение и $T_{gas}$ е температурата на разсейващия газ.

\subsection{Ранен ядрен синтез и ограничения следващи от него}
С напредъка на възрастта на вселената налягането и температурата са спадали и все още спадат постоянно. В периода на първата секунда след ГВ вече е било установено съотношението между протони и неутрони. Друго важно съотношение за образуването на частици и ядра в тази епоха е съотношението на бариони и фотони. Реакциите, които са отговаряли за формирането на протони и неутрони са изглеждали по следния начин:

\begin{gather}
  n + e^+  \rightleftharpoons p + \nu_e\\
  p + e^- \rightleftharpoons n + \Bar{\nu_e}
\end{gather}

където $n$ е неутрон, $p$ е протон, $e^-$ електрон, $e^+$ позитрон, $\nu_e$ е електронно неутрино, $\Bar{\nu_e}$ е електронно антинеутрино. Тези реакции са били в равновесие отначало, но с напредъка на времето температурата и плътността на материята са спаднали и реакциите са протичали с по-бавни темпове. Тъй като протона има малко по-малка маса/енергия и заради това съотношението $n/p$ също е спаднало с времето от 1 до $1/6$, когато е настъпил ``proton-neutron freeze out'' и съотношението е станало стабилно. Тогава обаче е добил значение факта, за разлика от протона, който е стабилен, неутрона не е и има период на полуразпад от $\approx 15$ мин. Поради това съотношението $n/p$ е продължило да спада. Само няколко минути след ГВ протоните са вече доминирали над неутроните. От това може да се извлече \textbf{едно от емпиричните ограничения}, които могат да се направят върху моделите за ранния ядрен синтез: тъй като наблюдаваме във вселената значително количество хелий (и поради това неутрони) \textbf{ранния ядрен синтез трябва да е протекъл в порядък от време близо до тези 15 мин}.\\

Ограничението от полуразпада на неутронните представлява горна граница за времето в което ядрен синтез е съществувал. Но може да се изчисли и \textbf{долна граница}. Тя е наложена от факта, че ранната вселена е била предимно доминирана от приноса на лъчението, не от барионната материя, тъмната материя или тъмната енергия. Последствие на тази доминация, е че първите ядра с повече от един нуклеон, създадени $\approx 10$ сек след ГВ, са били почти веднага \textbf{разцепени от силно енергетични фотони}. Тези първи ядра, са се състояли от протон и неутрон, защото това е била най-стабилната комбинация, и описаната реакция на синтез и разпад е изглеждала по следния начин:

\begin{gather}
  n + p  \rightleftharpoons \isotope[2][1]{H} + \gamma
\label{eq:deuterium}
\end{gather}

където $\gamma$ е гама фотон, а $\isotope[2][1]{H}$ е деутерий, изотоп на водорода. Доминацията на лъчението в тази фаза е от значение за развитието на вселената, защото ако не е съществувало ядреният синтез е щял да протече толкова бързо при наличната температура и налягане, че барионните да бъдат превърнати в много по-тежки елементи от водород и водорода да се изчерпи. Това би довело до много по-различно развитие на вселената след това, до много по-екстравагантни звезди и странен вид биологичен живот (ако въобще в такъв сценарий биха били възможни съществуването на звезди и живот). Въпреки началната доминация на лъчението с напредване на времето и разширяването на вселената енергията и плътността на фотоните е спаднала до толкова, че $\approx 90$ сек след ГВ лявата страна на ур. \eqref{eq:deuterium} е доминирала над дясната все повече и повече деутерий се е създавал. От там нататък по различни вериги са се образували и по-тежките изотопи на тритиум ($\isotope[3][1]{H}$), хелий-3 ($\isotope[3][2]{Hе}$) и в последствие и хелий ($\isotope[4][2]{Hе}$). Същото така някои от веригите за синтез на хелий са съдържали и литий и берилий. Някъде 15-20 мин след ГВ температурата и налягането във вселената са спаднали до толкова, че ядрен синтез не е могъл да бъде провеждан. Отношенията на елементарните ядра при края на синтеза са били приблизително (в количество ядра) 75\% водород, 25\% хелий и съвсем малки части деутерий, тритиум, хелий-3, литий и берилий. С това моделите твърдят, че елемнтарният състав на вселената днес е почти идентичен с този на вселената 20 мин след ГВ, тъй като по-тежките елементи създадени в супер нови по-късно, и които ни заобикалят сега, са незначително малки с космическото количество на водород и хелий в днешно време.


%%%%%%%%%%%%%%%%%%%%%%%%%%%%%%%%%%%%%%%%%%%%%%%%%
%%%%%%%%%%%%%%%%%%%%%%%%%%%%%%%%%%%%%%%%%%%%%%%%%%
\section*{Възможни въпроси за устна част}
\addcontentsline{toc}{section}{Възможни въпроси за устна част}

\begin{itemize}
\item Кои са различните видове поляризация и как се получават?
\item  Прилики и разлики на квазари и Сейферт галактики, радио галактики
\item  Сейферт: спирални, ярко ядро, силни емисии на многократно ионизирани елементи, широки емисионни линии от бързото въртене на горещия газ около ядрото; в сравнение с квазарите са по-малко ярки, по-близки от квазарите, визуално различими като галактики
\item блазари, по ярки от квазари, квазари, чиито джет е насочен срещу земята
\item  Естество на SN 1a
\item  граница на Чандрасекар и маса на бели джуджета
\item  ``ДХР, диаграми на сферични и разсеяни купове, да кажеш маса на бяло джудже, типове променливи звезди''
\item  морфология на галактики (Хъбъл класификация), популации, дисперсиите на скоростите, rotational support, възраст
\item  HI, HII райони
\item  стандартни свещи
\item  парадокс на Олберс
\end{itemize}


%%%%%%%%%%%%%%%%%%%%%%%%%%%%%%%%%%%%%%%%%%%%%%%%%%%%%%%%%%%%%%
%%%%%%%%%%%%%%%%%%%%%%%%%%%%%%%%%%%%%%%%%%%%%%%%%%%%%%%%%%
\section*{Извори}
\addcontentsline{toc}{section}{Извори}

\begin{itemize}
\item За начална функция на масите (IMF): \url{https://websites.pmc.ucsc.edu/~glatz/astr_112/lectures/notes19.pdf}
\item Ера на рекомбинация: \url{https://lweb.cfa.harvard.edu/~loeb/sciam.pdf}
\item Звездообразуване: \url{https://ned.ipac.caltech.edu/level5/Sept10/Grebel/Grebel_contents.html}
\item Галактики: \url{https://indico.cern.ch/event/927027/attachments/2104707/3540740/ASPlectures_Galaxies_Virtual_Sep2020_Part1_v2.pdf}
\item Учебници: \url{https://astro.phys.uni-sofia.bg/%D0%BC%D0%B0%D1%82%D0%B5%D1%80%D0%B8%D0%B0%D0%BB%D0%B8/}
\item ``Extragalactic Astronomy and Cosmology'', Peter Schneider, Springer 2006
\item разбираеми и човешки написани лекции: \url{http://astro.physics.uiowa.edu/~kaaret/s09/}
\item Други разбираеми лекции: \url{https://sites.ualberta.ca/~pogosyan/teaching/ASTRO_122/}
\item лекция за квазари и АГЯ: \url{https://www.slideshare.net/lwolberg/galaxies-quasars-lecture-notes-chapter-20}
\item Галактични купове и СЗЕ ефект: \url{https://sites.astro.caltech.edu/~george/ay127/Ay127_GalClusters.pdf}
\end{itemize}



\end{document}

